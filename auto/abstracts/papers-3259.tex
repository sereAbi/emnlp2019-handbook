Most state-of-the-art models for named entity recognition (NER) rely on the availability of large amounts of labeled data, making them challenging to extend to new, lower-resourced languages. However, there are now many proposed solutions to this problem involving either cross-lingual transfer learning, which learns from other highly resourced languages, or active learning, which efficiently selects effective training data based on model predictions. In this paper, we ask the question: given this recent progress, and some amount of human annotation, what is the most effective method for efficiently creating high-quality entity recognizers in under-resourced languages? Based on extensive experimentation using both simulated and real human annotation, we settle on a recipe of starting with a cross-lingual transferred model, then performing targeted annotation of only uncertain entity spans in the target language, minimizing annotator effort.
Results demonstrate that cross-lingual transfer is a powerful tool when very little data can be annotated, but an entity-targeted annotation strategy can achieve competitive accuracy quickly, with just one-tenth of training data.