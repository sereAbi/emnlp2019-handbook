Humor plays important role in human communication, which makes it important problem for natural language processing. Prior work on the analysis of humor focuses on whether text is humorous or not, or the degree of funniness, but this is insufficient to explain why it is funny. We therefore create a dataset on humor with 9,123 manually annotated jokes in Chinese. We propose a novel annotation scheme to give scenarios of how humor arises in text. Specifically, our annotations of linguistic humor not only contain the degree of funniness, like previous work, but they also contain key words that trigger humor as well as character relationship, scene, and humor categories. We report reasonable agreement between annotators. We also conduct an analysis and exploration of the dataset. To the best of our knowledge, we are the first to approach humor annotation for exploring the underlying mechanism of the use of humor, which may contribute to a significantly deeper analysis of humor. We also contribute with a scarce and valuable dataset, which we will release publicly.