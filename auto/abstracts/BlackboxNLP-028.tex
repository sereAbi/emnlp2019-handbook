Systematic compositionality is the ability to recombine meaningful units with regular and predictable outcomes, and it's seen as key to humans' capacity for generalization in language. Recent work (Lake and Baroni, 2018) has studied systematic compositionality in modern seq2seq models using generalization to novel navigation instructions in a grounded environment as a probing tool. Lake and Baroni's main experiment required the models to quickly bootstrap the meaning of new words. We extend this framework here to settings where the model needs only to recombine well-trained functional words (such as ``around'' and ``right'') in novel contexts. Our findings confirm and strengthen the earlier ones: seq2seq models can be impressively good at generalizing to novel combinations of previously-seen input, but only when they receive extensive training on the specific pattern to be generalized (e.g., generalizing from many examples of ``X around right'' to ``jump around right''), while failing when generalization requires novel application of compositional rules (e.g., inferring the meaning of ``around right'' from those of ``right'' and ``around'').
