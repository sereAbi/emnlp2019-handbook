Comments on web news contain controversies that manifest as inter-group agreement-conflicts. Tracking such rapidly evolving controversy may be used to ease conflict resolution and author-user interaction. However, this presupposes an online-learning prediction approach that scales to diverse domains using incidental supervision. To more deeply interpret comment-controversy decisions we frame prediction as binary classification and evaluate baselines and multi-task CNNs that use an auxiliary news-genre-encoder. Finally, we use ablation and interpretability methods to determine the impacts of topic/genre, discourse and sentiment indicators, contextual vs. global word influence, as well as genre-keywords vs. per-genre-controversy keywords -- to find that the models learn plausible controversy features using only incidentally supervised signals.
