Projecting linguistic annotations through word alignments is one of the most prevalent approaches to cross-lingual transfer learning. Conventional wisdom suggests that annotation projection ``just works'' regardless of the task at hand. We carefully consider multi-source projection for named entity recognition. Our experiment with 17 languages shows that to detect named entities in true low-resource languages, annotation projection may not be the right way to move forward. On a more positive note, we also uncover the conditions that do favor named entity projection from multiple sources. We argue these are infeasible under noisy low-resource constraints.
