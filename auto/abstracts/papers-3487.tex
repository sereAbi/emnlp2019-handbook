Beam search is universally used in (full-sentence) machine translation but its application to simultaneous translation remains highly non-trivial, where output words are committed on the fly. In particular, the recently proposed wait-k policy (Ma et al., 2018) is a simple and effective method that (after an initial wait) commits one output word on receiving each input word, making beam search seemingly inapplicable. To address this challenge, we propose a new speculative beam search algorithm that hallucinates several steps into the future in order to reach a more accurate decision by implicitly benefiting from a target language model. This idea makes beam search applicable for the first time to the generation of a single word in each step. Experiments over diverse language pairs show large improvement compared to previous work.