Ancient History relies on disciplines such as Epigraphy, the study of ancient inscribed texts, for evidence of the recorded past. However, these texts, "inscriptions", are often damaged over the centuries, and illegible parts of the text must be restored by specialists, known as epigraphists. This work presents Pythia, the first ancient text restoration model that recovers missing characters from a damaged text input using deep neural networks. Its architecture is carefully designed to handle long-term context information, and deal efficiently with missing or corrupted character and word representations. To train it, we wrote a non-trivial pipeline to convert PHI, the largest digital corpus of ancient Greek inscriptions, to machine actionable text, which we call PHI-ML. On PHI-ML, Pythia's predictions achieve a 30.1\% character error rate, compared to the 57.3\% of human epigraphists. Moreover, in 73.5\% of cases the ground-truth sequence was among the Top-20 hypotheses of Pythia, which effectively demonstrates the impact of this assistive method on the field of digital epigraphy, and sets the state-of-the-art in ancient text restoration.