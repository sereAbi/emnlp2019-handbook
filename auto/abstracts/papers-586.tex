Recent studies have significantly improved the state-of-the-art on common-sense reasoning (CSR) benchmarks like the Winograd Schema Challenge (WSC) and SWAG. The question we ask in this paper is whether improved performance on these benchmarks represents genuine progress towards common-sense-enabled systems. We make case studies of both benchmarks and design protocols that clarify and qualify the results of previous work by analyzing threats to the validity of previous experimental designs. Our protocols account for several properties prevalent in common-sense benchmarks including size limitations, structural regularities, and variable instance difficulty.