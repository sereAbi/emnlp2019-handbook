We describe a multi-agent communication framework for examining high-level linguistic phenomena at the community-level. We demonstrate that complex linguistic behavior observed in natural language can be reproduced in this simple setting: i) the outcome of contact between communities is a function of inter- and intra-group connectivity; ii) linguistic contact either converges to the majority protocol, or in balanced cases leads to novel creole languages of lower complexity; and iii) a linguistic continuum emerges where neighboring languages are more mutually intelligible than farther removed languages. We conclude that at least some of the intricate properties of language evolution need not depend on complex evolved linguistic capabilities, but can emerge from simple social exchanges between perceptually-enabled agents playing communication games.