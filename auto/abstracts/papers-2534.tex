Classical Chinese poetry is a jewel in the treasure house of Chinese culture.  Previous poem generation models only allow users to employ keywords to interfere the meaning of generated poems, leaving the dominion of generation to the model. In this paper, we propose a novel task of generating classical Chinese poems from vernacular, which allows users to have more control over the semantic of generated poems. We adapt the approach of unsupervised machine translation (UMT) to our task. We use segmentation-based padding and reinforcement learning to address under-translation and over-translation respectively. According to experiments, our approach significantly improve the perplexity and BLEU compared with typical UMT models. Furthermore, we explored guidelines on how to write the input vernacular to generate better poems. Human evaluation showed our approach can generate high-quality poems which are comparable to amateur poems.