Customers ask questions and customer service staffs answer their questions, which is the basic service model via multi-turn customer service (CS) dialogues on E-commerce platforms. Existing studies fail to provide comprehensive service satisfaction analysis, namely satisfaction polarity classification (e.g., well satisfied, met and unsatisfied) and sentimental utterance identification (e.g., positive, neutral and negative). In this paper, we conduct a pilot study on the task of service satisfaction analysis (SSA) based on multi-turn CS dialogues. We propose an extensible Context-Assisted Multiple Instance Learning (CAMIL) model to predict the sentiments of all the customer utterances and then aggregate those sentiments into service satisfaction polarity. After that, we propose a novel Context Clue Matching Mechanism (CCMM) to enhance the representations of all customer utterances with their matched context clues, i.e., sentiment and reasoning clues. We construct two CS dialogue datasets from a top E-commerce platform. Extensive experimental results are presented and contrasted against a few previous models to demonstrate the efficacy of our model.