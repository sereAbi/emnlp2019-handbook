Computational stylometry has become an increasingly important aspect of literary criticism, but many humanists lack the technical expertise or language-specific NLP resources required to exploit computational methods. We demonstrate a stylometry toolkit for analysis of Latin literary texts, which is freely available at www.qcrit.org/stylometry. Our toolkit generates data for a diverse range of literary features and has an intuitive point-and-click interface. The features included have proven effective for multiple literary studies and are calculated using custom heuristics without the need for syntactic parsing. As such, the toolkit models one approach to the user-friendly generation of stylometric data, which could be extended to other premodern and non-English languages  underserved by standard  NLP resources.