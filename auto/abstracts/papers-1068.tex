We propose a novel tensor embedding method that can effectively extract lexical features for humor recognition. Specifically, we use word-word co-occurrence to encode the contextual content of documents, and then decompose the tensor to get corresponding vector representations. We show that this simple method can capture features of lexical humor effectively for continuous humor recognition. In particular, we achieve a distance of 0.887 on a global humor ranking task, comparable to the top performing systems from SemEval 2017 Task 6B (Potash et al., 2017) but without the need for any external training corpus. In addition, we further show that this approach is also beneficial for small sample humor recognition tasks through a semi-supervised label propagation procedure, which achieves about 0.7 accuracy on the 16000 One-Liners (Mihalcea and Strapparava, 2005) and Pun of the Day (Yang et al., 2015) humour classification
datasets using only 10\% of known labels.