Distance supervision is widely used in relation extraction tasks, particularly when large-scale manual annotations are virtually impossible to conduct. Although Distantly Supervised Relation Extraction (DSRE) benefits from automatic labelling, it suffers from serious mislabelling issues, i.e. some or all of the instances for an entity pair (head and tail entities) do not express the labelled relation. In this paper, we propose a novel model that employs a collaborative curriculum learning framework to reduce the effects of mislabelled data. Specifically, we firstly propose an internal self-attention mechanism between the convolution operations in convolutional neural networks (CNNs) to learn a better sentence representation from the noisy inputs. Then we define two sentence selection models as two relation extractors in order to collaboratively learn and regularise each other under a curriculum scheme to alleviate noisy effects, where the curriculum could be constructed by conflicts or small loss. Finally, experiments are conducted on a widely-used public dataset and the results indicate that the proposed model significantly outperforms baselines including the state-of-the-art in terms of P\@N and PR curve metrics, thus evidencing its capability of reducing noisy effects for DSRE.