Humor is a unique and creative communicative behavior often displayed during social interactions. It is produced in a multimodal manner, through the usage of words (text), gestures (visual) and prosodic cues (acoustic). Understanding humor from these three modalities falls within boundaries of multimodal language; a recent research trend in natural language processing that models natural language as it happens in face-to-face communication. Although humor detection is an established research area in NLP, in a multimodal context it has been understudied. This paper presents a diverse multimodal dataset, called UR-FUNNY, to open the door to understanding multimodal language used in expressing humor. The dataset and accompanying studies, present a framework in multimodal humor detection for the natural language processing community. UR-FUNNY is publicly available for research.