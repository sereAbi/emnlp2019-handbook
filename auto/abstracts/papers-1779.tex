In multi-document summarization, a set of documents to be summarized is assumed to be on the same topic, known as the underlying topic in this paper. That is, the underlying topic can be collectively represented by all the documents in the set. Meanwhile, different documents may cover various different subtopics and the same subtopic can be across several documents. Inspired by topic model, the underlying topic of a document set can also be viewed as a collection of different subtopics of different importance. In this paper, we propose a summarization model called STDS. The model generates the underlying topic representation from both document view and subtopic view in parallel. The learning objective is to minimize the distance between the representations learned from the two views. The contextual information is encoded through a hierarchical RNN architecture. Sentence salience is estimated in a hierarchical way with subtopic salience and relative sentence salience, by considering the contextual information. Top ranked  sentences are then extracted as a summary. Note that the notion of subtopic enables us to bring in additional information (e.g. comments to news articles) that is helpful for document summarization. Experimental results show that the proposed solution outperforms state-of-the-art methods on benchmark datasets.