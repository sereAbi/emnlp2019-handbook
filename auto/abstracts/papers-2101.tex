Elazar and Goldberg (2018) showed that protected attributes can be extracted from the representations of a debiased neural network for mention detection at above-chance levels,  by evaluating a diagnostic classifier on a held-out subsample of the data it was trained on. We revisit their experiments and conduct a series of follow-up experiments showing that, in fact, the diagnostic  classifier generalizes poorly to both new in-domain samples and new domains, indicating that it relies on correlations specific to their particular data sample. We further show that a diagnostic classifier trained on the biased baseline neural network also does not generalize to new samples. In other words, the biases detected in Elazar and Goldberg (2018) seem restricted to their particular data sample, and would therefore not bias the decisions of the model on new samples, whether in-domain or out-of-domain. In light of this, we discuss better methodologies for detecting bias in our models.