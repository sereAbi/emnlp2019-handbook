Supervised learning models often perform poorly at low-shot tasks, i.e. tasks for which little labeled data is available for training. One prominent approach for improving low-shot learning is to use unsupervised pre-trained neural models. Another approach is to obtain richer supervision by collecting annotator rationales (explanations supporting label annotations). In this work, we combine these two approaches to improve low-shot text classification with two novel methods: a simple bag-of-words embedding approach; and a more complex context-aware method, based on the BERT model. In experiments with two English text classification datasets, we demonstrate substantial performance gains from combining pre-training with rationales. Furthermore, our investigation of a range of train-set sizes reveals that the simple bag-of-words approach is the clear top performer when there are only a few dozen training instances or less, while more complex models, such as BERT or CNN, require more training data to shine.