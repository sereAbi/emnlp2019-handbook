Intent classification is an important building block of dialogue systems. With the burgeoning of conversational AI, existing systems are not capable of handling numerous fast-emerging intents, which motivates zero-shot intent classification. Nevertheless, research on this problem is still in the incipient stage and few methods are available. A recently proposed zero-shot intent classification method, IntentCapsNet, has been shown to achieve state-of-the-art performance. However, it has two unaddressed limitations: (1) it cannot deal with polysemy when extracting semantic capsules; (2) it hardly recognizes the utterances of unseen intents in the generalized zero-shot intent classification setting. To overcome these limitations, we propose to reconstruct capsule networks for zero-shot intent classification. First, we introduce a dimensional attention mechanism to fight against polysemy. Second, we reconstruct the transformation matrices for unseen intents by utilizing abundant latent information of the labeled utterances, which significantly improves the model generalization ability. Experimental results on two task-oriented dialogue datasets in different languages show that our proposed method outperforms IntentCapsNet and other strong baselines.