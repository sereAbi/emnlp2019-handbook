Despite their practical success and impressive performances, neural-network-based and distributed semantics techniques have often been criticized as they remain fundamentally opaque and difficult to interpret. In a vein similar to recent pieces of work investigating the linguistic abilities of these representations, we study another core, defining property of language: the property of long-distance dependencies. Human languages exhibit the ability to interpret discontinuous elements distant from each other in the string as if they were adjacent. This ability is blocked if a similar, but extraneous, element intervenes between the discontinuous components. We present results that show, under exhaustive and precise conditions, that one kind of word embeddings and the similarity spaces they define do not encode the properties of intervention similarity in long-distance dependencies, and that therefore they fail to represent this core linguistic notion.
