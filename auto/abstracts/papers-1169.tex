Text attribute transfer aims to automatically rewrite sentences such that they possess certain linguistic attributes, while simultaneously preserving their semantic content. This task remains challenging due to a  lack of supervised parallel data. Existing approaches try to explicitly disentangle content and attribute information, but this is difficult and often results in poor content-preservation and ungrammaticality. In contrast, we propose a simpler approach, Iterative Matching and Translation (IMaT), which: (1) constructs a pseudo-parallel corpus by aligning a subset of semantically similar sentences from the source and the target corpora; (2) applies a standard sequence-to-sequence model to learn the attribute transfer; (3) iteratively improves the learned transfer function by refining imperfections in the alignment. In sentiment modification and formality transfer tasks, our  method outperforms complex state-of-the-art systems by a large margin. As an auxiliary contribution, we produce a publicly-available test set with human-generated transfer references.