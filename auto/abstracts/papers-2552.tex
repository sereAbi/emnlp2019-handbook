Byte-Pair Encoding (BPE) is an unsupervised sub-word tokenization technique, commonly used in neural machine translation and other NLP tasks.
Its effectiveness makes it a de facto standard, but the reasons for this are not well understood.
We link BPE to the broader family of dictionary-based compression algorithms and compare it with other members of this family.
Our experiments across datasets, language pairs, translation models, and vocabulary size show that - given a fixed vocabulary size budget - the fewer tokens an algorithm needs to cover the test set, the better the translation (as measured by BLEU).