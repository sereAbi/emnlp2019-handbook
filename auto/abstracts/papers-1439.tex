Causal interpretation of correlational findings from observational studies has been a major type of misinformation in science communication. Prior studies on identifying inappropriate use of causal language relied on manual content analysis, which is not scalable for examining a large volume of science publications. In this study, we first annotated a corpus of over 3,000 PubMed research conclusion sentences, then developed a BERT-based prediction model that classifies conclusion sentences into “no relationship”, “correlational”, “conditional causal”, and “direct causal” categories, achieving an accuracy of 0.90 and a macro-F1 of 0.88. We then applied the prediction model to measure the causal language use in the research conclusions of about 38,000 observational studies in PubMed. The prediction result shows that 21.7\% studies used direct causal language exclusively in their conclusions, and 32.4\% used some direct causal language. We also found that the ratio of causal language use differs among authors from different countries, challenging the notion of a shared consensus on causal language use in the global science community. Our prediction model could also be used to help identify the inappropriate use of causal language in science publications.