The increasing suicide rates amongst youth and its high correlation with suicidal ideation expression on social media warrants a deeper investigation into models for the detection of suicidal intent in text such as tweets to enable prevention. However, the complexity of the natural language constructs makes this task very challenging. Deep Learning architectures such as LSTMs, CNNs, and RNNs show promise in sentence level classification problems. This work investigates the ability of deep learning architectures to build an accurate and robust model for suicidal ideation detection and compares their performance with standard baselines in text classification problems. The experimental results reveal the merit in C-LSTM based models compared to other deep learning and machine learning based classification models for suicidal ideation detection.
