With the growth of the internet, the number of fake-news online has been proliferating every year. The consequences of such phenomena are manifold, ranging from lousy decision-making process to bullying and violence episodes. Therefore, fact-checking algorithms became a valuable asset. To this aim, an important step to detect fake-news is to have access to a credibility score for a given information source. However, most of the widely used Web indicators have either been shut-down to the public (e.g., Google PageRank) or are not free for use (Alexa Rank). Further existing databases are short-manually curated lists of online sources, which do not scale. Finally, most of the research on the topic is theoretical-based or explore confidential data in a restricted simulation environment. In this paper we explore current research, highlight the challenges and propose solutions to tackle the problem of classifying websites into a credibility scale. The proposed model automatically extracts source reputation cues and computes a credibility factor, providing valuable insights which can help in belittling dubious and confirming trustful unknown websites. Experimental results outperform state of the art in the 2-classes and 5-classes setting.
