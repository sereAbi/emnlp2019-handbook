Replacing static word embeddings with contextualized word representations has yielded significant improvements on many NLP tasks. However, just how contextual are the contextualized representations produced by models such as ELMo and BERT? Are there infinitely many context-specific representations for each word, or are words essentially assigned one of a finite number of word-sense representations? For one, we find that the contextualized representations of all words are not isotropic in any layer of the contextualizing model. While representations of the same word in different contexts still have a greater cosine similarity than those of two different words, this self-similarity is much lower in upper layers. This suggests that upper layers of contextualizing models produce more context-specific representations, much like how upper layers of LSTMs produce more task-specific representations. In all layers of ELMo, BERT, and GPT-2, on average, less than 5\% of the variance in a word's contextualized representations can be explained by a static embedding for that word, providing some justification for the success of contextualized representations.