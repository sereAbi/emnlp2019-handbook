This paper describes the changes applied to the original process used to convert the Index Thomisticus Treebank, a corpus including texts in Medieval Latin by Thomas Aquinas, into the annotation style of Universal Dependencies. The changes are made both to harmonise the Universal Dependencies version of the Index Thomisticus Treebank with the two other available Latin treebanks and to fix errors and inconsistencies resulting from the original process. The paper details the treatment of different issues in PoS tagging, lemmatisation and assignment of dependency relations. Finally, it assesses the quality of the new conversion process by providing an evaluation against a gold standard.
