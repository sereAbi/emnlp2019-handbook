Training and testing many possible parameters or model architectures of state-of-the-art machine translation or automatic speech recognition system is a cumbersome task. They usually require a long pipeline of commands reaching from pre-processing the training data to post-processing and evaluating the output. This paper introduces Sisyphus, a tool that aims at managing scientific experiments in an efficient way. After defining the workflow for a given task, Sisyphus runs all required steps and ensures that all commands finish successfully. It avoids unnecessary computations by reusing tasks that are needed for multiple parts of the workflow and saves the user time by determining the order in which the tasks are to be performed. Since the program and workflow are written in Python they can be easily extended to contain arbitrary code. This makes it possible to use the rich collection of Python tools for editing, debugging, and documentation. It only has few requirements on the underlying server or cluster, and has been successfully tested in many large scale setups and can handle thousands of tasks inside the workflow.
