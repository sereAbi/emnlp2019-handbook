Computing author intent from multimodal data like Instagram posts requires modeling a complex relationship between text and image. For example, a caption might evoke an ironic contrast with the image, so neither caption nor image is a mere transcript of the other. Instead they combine---via what has been called meaning multiplication (Bateman et al.)- to create a new meaning that has a more complex relation to the literal meanings of text and image. Here we introduce a multimodal dataset of \$1299\$ Instagram posts labeled for three orthogonal taxonomies: the authorial intent behind the image-caption pair, the contextual relationship between the literal meanings of the image and caption, and the semiotic relationship between the signified meanings of the image and caption. We build a baseline deep multimodal classifier to validate the taxonomy, showing that employing both text and image improves intent detection by 9.6 compared to using only the image modality, demonstrating the commonality of non-intersective meaning multiplication. The gain with multimodality is greatest when the image and caption diverge semiotically. Our dataset offers a new resource for the study of the rich meanings that result from pairing text and image.