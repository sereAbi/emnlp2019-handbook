Crowdsourcing has been the prevalent paradigm for creating natural language understanding datasets in recent years. A common crowdsourcing practice is to recruit a small number of high-quality workers, and have them massively generate examples. Having only a few workers generate the majority of examples raises concerns about data diversity, especially when workers freely generate sentences. In this paper, we perform a series of experiments showing these concerns are evident in three recent NLP datasets. We show that model performance improves when training with annotator identifiers as features, and that models are able to recognize the most productive annotators. Moreover, we show that often models do not generalize well to examples from annotators that did not contribute to the training set. Our findings suggest that annotator bias should be monitored during dataset creation, and that test set annotators should be disjoint from training set annotators.