Semantic role labeling (SRL) involves extracting propositions (i.e. predicates and their typed arguments) from natural language sentences. State-of-the-art SRL models rely on powerful encoders (e.g., LSTMs) and do not model non-local interaction between arguments. We propose a new approach to modeling these interactions while maintaining efficient inference. Specifically, we use  Capsule Networks (Sabour et al., 2017): each proposition is encoded as a tuple of  \textit{capsules}, one capsule per argument type (i.e. role). These tuples serve as embeddings of entire propositions.   In every network layer, the capsules interact with each other and with representations of words in the sentence.  Each iteration results in updated proposition embeddings and updated predictions about the SRL structure. Our model substantially outperforms the non-refinement baseline model on all 7 CoNLL-2019 languages and achieves state-of-the-art results on 5 languages (including English) for dependency SRL. We analyze the types of mistakes corrected by the refinement procedure. For example,  each role is typically (but not always) filled with at most one argument. Whereas enforcing this approximate constraint is not useful with the modern SRL system, iterative procedure corrects the mistakes by capturing this intuition in a flexible and context-sensitive way.