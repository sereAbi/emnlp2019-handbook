Semantic parsing aims to map natural language utterances onto machine interpretable meaning representations, aka programs whose execution against a real-world environment produces a denotation. Weakly-supervised semantic parsers are trained on utterance-denotation pairs treating programs as latent. The task is challenging due to the large search space and spuriousness of programs which may execute to the correct answer but do not generalize to unseen examples.  Our goal is to instill an inductive bias in the parser to help it distinguish between spurious and correct programs.  We capitalize on the intuition that correct programs would likely respect certain structural constraints were they to be aligned to the question (e.g., program fragments are unlikely to align to overlapping text spans) and propose to model alignments as structured latent variables.  In order to make the latent-alignment framework tractable, we decompose the parsing task into (1) predicting a partial ``abstract program'' and (2) refining it while modeling structured alignments with differential dynamic programming. We obtain state-of-the-art performance on the WikiTableQuestions and WikiSQL datasets. When compared to a standard attention baseline, we observe that the proposed structured-alignment mechanism is highly beneficial.