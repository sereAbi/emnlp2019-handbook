A core problem of information retrieval (IR) is relevance matching, which is to rank documents by relevance to a user's query. On the other hand, many NLP problems, such as question answering and paraphrase identification, can be considered variants of semantic matching, which is to measure the semantic distance between two pieces of short texts. 
While at a high level both relevance and semantic matching require modeling textual similarity, many existing techniques for one cannot be easily adapted to the other. To bridge this gap, we propose a novel model, HCAN{\textasciitilde}(Hybrid Co-Attention Network),  that comprises (1) a hybrid encoder module that includes ConvNet-based and LSTM-based encoders, (2) a relevance matching module that measures soft term matches with importance weighting at multiple granularities, and (3) a semantic matching module with co-attention mechanisms that capture context-aware semantic relatedness. Evaluations on multiple IR and NLP benchmarks demonstrate state-of-the-art effectiveness compared to approaches that do not exploit pretraining on external data. Extensive ablation studies suggest that relevance and semantic matching signals are complementary across many problem settings, regardless of the choice of underlying encoders.