Programmers typically organize executable source code using high-level coding patterns or idiomatic structures such as nested loops, exception handlers and recursive blocks, rather than as individual code tokens. In contrast, state of the art (SOTA) semantic parsers still map natural language instructions to source code by building the code syntax tree one node at a time. In this paper, we introduce an iterative method to extract code idioms from large source code corpora by repeatedly collapsing most-frequent depth-2 subtrees of their syntax trees, and train semantic parsers to apply these idioms during decoding. Applying idiom-based decoding on a recent context-dependent semantic parsing task improves the SOTA by $2.2\%$ BLEU score while reducing training time by more than $50\%$. This improved speed enables us to scale up the model by training on an extended training set that is 5 times larger, to further move up the SOTA by an additional $2.3\%$ BLEU and $0.9\%$ exact match. Finally, idioms also significantly improve accuracy of semantic parsing to SQL on the ATIS-SQL dataset, when training data is limited.