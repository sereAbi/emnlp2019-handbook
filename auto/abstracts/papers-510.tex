A significant barrier to progress in data-driven approaches to building dialog systems is the lack of high quality, goal-oriented conversational data. To help satisfy this elementary requirement, we introduce the initial release of the Taskmaster-1 dataset which includes 13,215 task-based dialogs comprising six domains. Two procedures were used to create this collection, each with unique advantages. The first involves a two-person, spoken ``Wizard of Oz'' (WOz) approach in which trained agents and crowdsourced workers interact to complete the task while the second is self-dialog in which crowdsourced workers write the entire dialog themselves. We do not restrict the workers to detailed scripts or to a small knowledge base and hence we observe that our dataset contains more realistic and diverse conversations in comparison to existing datasets. We offer several baseline models including state of the art neural seq2seq architectures with benchmark performance as well as qualitative human evaluations. Dialogs are labeled with API calls and arguments, a simple and cost effective approach which avoids the requirement of complex annotation schema. The layer of abstraction between the dialog model and the service provider API allows for a given model to interact with multiple services that provide similar functionally. Finally, the dataset will evoke interest in written vs. spoken language, discourse patterns, error handling and other linguistic phenomena related to dialog system research, development and design.
