Recent efforts in cross-lingual word embedding (CLWE) learning have predominantly focused on fully unsupervised approaches that project monolingual embeddings into a shared cross-lingual space without any cross-lingual signal. The lack of any supervision makes such approaches conceptually attractive. Yet, their only core difference from (weakly) supervised projection-based CLWE methods is in the way they obtain a seed dictionary used to initialize an iterative self-learning procedure. The fully unsupervised methods have arguably become more robust, and their primary use case is CLWE induction for pairs of resource-poor and distant languages. In this paper, we question the ability of even the most robust unsupervised CLWE approaches to induce meaningful CLWEs in these more challenging settings. A series of bilingual lexicon induction (BLI) experiments with 15 diverse languages (210 language pairs) show that fully unsupervised CLWE methods still fail for a large number of language pairs (e.g., they yield zero BLI performance for 87/210 pairs). Even when they succeed, they never surpass the performance of weakly supervised methods (seeded with 500-1,000 translation pairs) using the same self-learning procedure in any BLI setup, and the gaps are often substantial. These findings call for revisiting the main motivations behind fully unsupervised CLWE methods.