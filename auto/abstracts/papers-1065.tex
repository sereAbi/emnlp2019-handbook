Human translators routinely have to translate rare inflections of words -- due to the Zipfian distribution of words in a language. When translating from Spanish, a good translator would have no problem identifying the proper translation of a statistically rare inflection such as habláramos. Note the lexeme itself, hablar, is relatively common. In this work, we investigate whether state-of-the-art bilingual lexicon inducers are capable of learning this kind of generalization. We introduce 40 morphologically complete dictionaries in 10 languages and evaluate three of the best performing models on the task of translation of less frequent morphological forms. We demonstrate that the performance of state-of-the-art models drops considerably when evaluated on infrequent morphological inflections and then show that adding a simple morphological constraint at training time improves the performance, proving that the bilingual lexicon inducers can benefit from better encoding of morphology.