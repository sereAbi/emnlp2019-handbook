The recent explosion of false claims in social media and on the Web in general has given rise to a lot of manual fact-checking initiatives. Unfortunately, the number of claims that need to be fact-checked is several orders of magnitude larger than what humans can handle manually. Thus, there has been a lot of research aiming at automating the process.  Interestingly, previous work has largely ignored the growing number of claims about images. This is despite the fact that visual imagery is more influential than text and naturally appears alongside fake news. Here we aim at bridging this gap. In particular, we create a new dataset for this problem, and we explore a variety of features modeling the claim, the image, and the relationship between the claim and the image. The evaluation results show sizable improvements over the baseline. We release our dataset, hoping to enable further research on fact-checking claims about images.