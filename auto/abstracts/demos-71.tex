Dialogue systems have the potential to change how  people  interact  with  machines  but  are highly  dependent  on  the  quality  of  the  data used  to  train  them.It  is  therefore  important to develop good dialogue annotation tools which can improve the speed and quality of dialogue data annotation.  With this in mind, we introduce LIDA, an annotation tool designed specifically  for  conversation  data.   As  far  as we know, LIDA is the first dialogue annotation system that handles the entire dialogue annotation  pipeline  from  raw  text,  as  may  be  the output of transcription services,  to structured conversation data. Furthermore it supports the integration of arbitrary machine learning mod-els as annotation recommenders and also has a dedicated interface to resolve inter-annotator disagreements such as after crowdsourcing an-notations  for  a  dataset.   LIDA  is  fully  open source, documented and publicly available.[https://github.com/Wluper/lida]