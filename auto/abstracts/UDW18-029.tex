The Hebrew treebank (HTB),  consisting of  6221 morpho-syntactically annotated newspaper sentences, has been  the  only resource  for training and validating  Hebrew statistical  parsers  for almost two decades now. During these decades, the HTB  has  gone through a  trajectory  of automatic and semi-automatic conversions, until arriving at its  current UDv2 form. In this work we set out to manually validate the  UDv2 version and, accordingly,  we apply  scheme changes  to bring the UD HTB into the same theoretical ground as the rest of UD. Our experimental results show that improving the linguistic coherence and internal consistency of the UD HTB has indeed  led to improved  syntactic parsing  performance. At the same time, there is  more to be done at the points of intersection  with other linguistic processing layers, in particular, at the interface of UD with external morphological and lexical resources.
