Memory neurons of long short-term memory (LSTM) networks encode and process information in powerful yet mysterious ways. While there has been work to analyze their behavior in carrying low-level information such as linguistic properties, how they directly contribute to label prediction remains unclear. We find inspiration from biologists and study the affinity between individual neurons and labels, propose a novel metric to quantify the sensitivity of neurons to each label, and conduct experiments to show the validity of our proposed metric. We discover that some neurons are trained to specialize on a subset of labels, and while dropping an arbitrary neuron has little effect on the overall accuracy of the model, dropping label-specialized neurons predictably and significantly degrades prediction accuracy on the associated label. We further examine the consistency of neuron-label affinity across different models. These observations provide insight into the inner mechanisms of LSTMs.