Neural language models are usually trained using Maximum-Likelihood Estimation (MLE). The corresponding objective function for MLE is derived from the Kullback-Leibler (KL) divergence between the empirical probability distribution representing the data and the parametric probability distribution output by the model. However, the word frequency discrepancies in natural language make performance extremely uneven: while the perplexity is usually very low for frequent words, it is especially difficult to predict rare words. In this paper, we experiment with several families (alpha, beta and gamma) of power divergences, generalized from the KL divergence, for learning language models with an objective different than standard MLE. Intuitively, these divergences should affect the way the probability mass is spread during learning, notably by prioritizing performances on high or low-frequency words. In addition, we implement and experiment with various sampling-based objectives, where the computation of the output layer is only done on a small subset of the vocabulary. They are derived as power generalizations of a softmax approximated via Importance Sampling, and Noise Contrastive Estimation, for accelerated learning. Our experiments on the Penn Treebank and Wikitext-2 show that these power divergences can indeed be used to prioritize learning on the frequent or rare words, and lead to general performance improvements in the case of sampling-based learning.