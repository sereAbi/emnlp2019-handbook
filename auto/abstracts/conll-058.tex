We investigate the relation between the transposition and deletion effects in word reading, i.e., the finding that readers can successfully read ``SLAT'' as ``SALT'', or ``WRK'' as ``WORK'', and the neighborhood effect. In particular, we investigate whether lexical orthographic neighborhoods take into account transposition and deletion in determining neighbors. If this is the case, it is more likely that the neighborhood effect takes place early during processing, and does not solely rely on similarity of internal representations. We introduce a new neighborhood measure, rd20, which can be used to quantify neighborhood effects over arbitrary feature spaces. We calculate the rd20 over large sets of words in three languages using various feature sets and show that feature sets that do not allow for transposition or deletion explain more variance in Reaction Time (RT) measurements. We also show that the rd20 can be calculated using the hidden state representations of an Multi-Layer Perceptron, and show that these explain less variance than the raw features. We conclude that the neighborhood effect is unlikely to have a perceptual basis, but is more likely to be the result of items co-activating after recognition. All code is available at: \url{www.github.com/clips/conll2018}
