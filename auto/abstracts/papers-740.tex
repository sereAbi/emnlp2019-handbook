The incorporation of pseudo data in the training of grammatical error correction models has been one of the main factors in improving the performance of such models.
However, consensus is lacking on experimental configurations, namely, choosing how the pseudo data should be generated or used. 
In this study, these choices are investigated through extensive experiments, and state-of-the-art performance is achieved on the CoNLL-2014 test set (F0.5=65.0) and the official test set of the BEA-2019 shared task (F0.5=70.2) without making any modifications to the model architecture.