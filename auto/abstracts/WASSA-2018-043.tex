In this paper, automatic homophone- and homograph detection are suggested as new useful features for humor recognition systems.  The system combines style-features from previous studies on humor recognition in short text with ambiguity-based features. The performance of two potentially useful homograph detection methods is evaluated using crowdsourced annotations as ground truth. Adding homophones and homographs as features to the classifier results in a small but significant improvement over the style-features alone. For the task of humor recognition, recall appears to be a more important quality measure than precision. Although the system was designed for humor recognition in oneliners, it also performs well at the classification of longer humorous texts.
