In natural language inference (NLI), contexts are considered veridical if they allow us to infer that their underlying propositions make true claims about the real world. We investigate whether a state-of-the-art natural language inference model (BERT) learns to make correct inferences about veridicality in verb-complement constructions. We introduce an NLI dataset for veridicality evaluation consisting of 1,500 sentence pairs, covering 137 unique verbs. We find that both human and model inferences generally follow theoretical patterns, but exhibit a systematic bias towards assuming that verbs are veridical--a bias which is amplified in BERT. We further show that, encouragingly, BERT's inferences are sensitive not only to the presence of individual verb types, but also to the syntactic role of the verb, the form of the complement clause (to- vs. that-complements), and negation.