We introduce Morse, a recurrent encoder-decoder model that produces morphological analyses of each word in a sentence.  The encoder turns the relevant information about the word and its context into a fixed size vector representation and the decoder generates the sequence of characters for the lemma followed by a sequence of individual morphological features. We show that generating morphological features individually rather than as a combined tag allows the model to handle rare or unseen tags and outperform whole-tag models. In addition, generating morphological features as a sequence rather than e.g. an unordered set allows our model to produce an arbitrary number of features that represent multiple inflectional groups in morphologically complex languages. We obtain state-of-the art results in nine languages of different morphological complexity under low-resource, high-resource and transfer learning settings. We also introduce TrMor2018, a new high accuracy Turkish morphology dataset. Our Morse implementation and the TrMor2018 dataset are available online to support future research.