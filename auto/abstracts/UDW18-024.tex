We describe and evaluate different approaches to the conversion of gold standard corpus data from Stanford Typed Dependencies (SD) and Penn-style constituent trees to the latest English Universal Dependencies representation (UD 2.2). Our results indicate that pure SD to UD conversion is highly accurate across multiple genres, resulting in around 1.5\% errors, but can be improved further to fewer than 0.5\% errors given access to annotations beyond the pure syntax tree, such as entity types and coreference resolution, which are necessary for correct generation of several UD relations. We show that constituent-based conversion using CoreNLP (with automatic NER) performs substantially worse in all genres, including when using gold constituent trees, primarily due to under-specification of phrasal grammatical functions.
