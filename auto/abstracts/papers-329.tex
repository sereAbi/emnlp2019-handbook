Multi-hop knowledge graph (KG) reasoning is an effective and explainable method for predicting the target entity via reasoning paths in query answering (QA) task. Most previous methods assume that every relation in KGs has enough triples for training, regardless of those few-shot relations which cannot provide sufficient triples for training robust reasoning models. In fact, the performance of existing multi-hop reasoning methods drops significantly on few-shot relations. In this paper, we propose a meta-based multi-hop reasoning method (Meta-KGR), which adopts meta-learning to learn effective meta parameters from high-frequency relations that could quickly adapt to few-shot relations. We evaluate Meta-KGR on two public datasets sampled from Freebase and NELL, and the experimental results show that Meta-KGR outperforms state-of-the-art methods in few-shot scenarios. In the future, our codes and datasets will also be available to provide more details.