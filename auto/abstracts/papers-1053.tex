Previous research on empathetic dialogue systems has mostly focused on generating responses given certain emotions. However, being empathetic not only requires the ability of generating emotional responses, but more importantly, requires the understanding of user emotions and replying appropriately. In this paper, we propose a novel end-to-end approach for modeling empathy in dialogue systems: Mixture of Empathetic Listeners (MoEL). Our model first captures the user emotions and outputs an emotion distribution. Based on this, MoEL will softly combine the output states of the appropriate Listener(s), which are each optimized to react to certain emotions, and generate an empathetic response.
Human evaluations on EMPATHETIC-DIALOGUES dataset confirm that MoEL outperforms multitask training baseline in terms of empathy, relevance, and fluency. Furthermore, the case study on generated responses of different Listeners shows high interpretability of our model.