Active learning (AL) is a widely-used training strategy for maximizing predictive performance subject to a fixed annotation budget. In AL, one iteratively selects training examples for annotation, often those for which the current model is most uncertain (by some measure). The hope is that active sampling leads to better performance than would be achieved under independent and identically distributed (i.i.d.) random samples. While AL has shown promise in retrospective evaluations, these studies often ignore practical obstacles to its use. In this paper, we show that while AL may provide benefits when used with specific models and for particular domains, the benefits of current approaches do not generalize reliably across models and tasks. This is problematic because in practice, one does not have the opportunity to explore and compare alternative AL strategies. Moreover, AL couples the training dataset with the model used to guide its acquisition. We find that subsequently training a successor model with an actively-acquired dataset does not consistently outperform training on i.i.d. sampled data. Our findings raise the question of whether the downsides inherent to AL are worth the modest and inconsistent performance gains it tends to afford.