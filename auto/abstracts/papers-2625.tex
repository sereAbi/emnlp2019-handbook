Discourse parsing could not yet take full advantage of the neural NLP revolution, mostly due to the lack of annotated datasets. We propose a novel approach that uses distant supervision on an auxiliary task (sentiment classification), to generate abundant data for RST-style discourse structure prediction. Our approach combines a neural variant of multiple-instance learning, using document-level supervision, with an optimal CKY-style tree generation algorithm. 
In a series of experiments, we train a discourse parser (for only structure prediction) on our automatically generated dataset and compare it with parsers trained on human-annotated corpora (news domain RST-DT and Instructional domain). Results indicate that while our parser does not yet match the performance of a parser trained and tested on the same dataset (intra-domain), it does perform remarkably well on the much more difficult and arguably more useful task of inter-domain discourse structure prediction, where the parser is trained on one domain and tested/applied on another one.