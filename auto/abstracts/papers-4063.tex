Probes, supervised models trained to predict properties (like parts-of-speech) from representations (like ELMo), have achieved high accuracy on a range of linguistic tasks. But does this mean that the representations encode linguistic structure or just that the probe has learned the linguistic task? In this paper, we propose control tasks, which associate word types with random outputs, to complement linguistic tasks. By construction, these tasks can only be learned by the probe itself. So a good probe, (one that reflects the representation), should be selective, achieving high linguistic task accuracy and low control task accuracy. The selectivity of a probe puts linguistic task accuracy in context with the probe's capacity to memorize from word types. We construct control tasks for English part-of-speech tagging and dependency edge prediction, and show that popular probes on ELMo representations are not selective. We also find that dropout, commonly used to control probe complexity, is ineffective for improving selectivity of MLPs, but that other forms of regularization are effective. Finally, we find that while probes on the first layer of ELMo yield slightly better part-of-speech tagging accuracy than the second, probes on the second layer are substantially more selective, which raises the question of which layer better represents parts-of-speech.