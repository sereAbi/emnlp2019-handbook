Phonetic similarity algorithms identify words and phrases with similar pronunciation which are used in many natural language processing tasks. However, existing approaches are designed mainly for Indo-European languages and fail to capture the unique properties of Chinese pronunciation. In this paper, we propose a high dimensional encoded phonetic similarity algorithm for Chinese, DIMSIM. The encodings are learned from annotated data to separately map initial and final phonemes into n-dimensional coordinates. Pinyin phonetic similarities are then calculated by aggregating the similarities of initial, final and tone. DIMSIM demonstrates a 7.5X improvement on mean reciprocal rank over the state-of-the-art phonetic similarity approaches.
