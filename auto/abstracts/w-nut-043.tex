In this paper, we provide a lexical comparative analysis of the vocabulary used by customers and agents in an Enterprise Resource Planning (ERP) environment and a potential solution to clean the data and extract relevant content for NLP. As a result, we demonstrate that the actual vocabulary for the language that prevails in the ERP conversations is highly divergent from the standardized dictionary and further different from general language usage as extracted from the Common Crawl corpus. Moreover, in specific business communication circumstances, where it is expected to observe a high usage of standardized language, code switching and non-standard expression are predominant, emphasizing once more the discrepancy between the day-to-day use of language and the standardized one.
