Text-based Question Generation (QG) aims at generating natural and relevant questions that can be answered by a given answer in some context. Existing QG models suffer from a “semantic drift” problem, i.e., the semantics of the model-generated question drifts away from the given context and answer. In this paper, we first propose two semantics-enhanced rewards obtained from downstream question paraphrasing and question answering tasks to regularize the QG model to generate semantically valid questions. Second, since the traditional evaluation metrics (e.g., BLEU) often fall short in evaluating the quality of generated questions, we propose a QA-based evaluation method which measures the QG model’s ability to mimic human annotators in generating QA training data. Experiments show that our method achieves the new state-of-the-art performance w.r.t. traditional metrics, and also performs best on our QA-based evaluation metrics. Further, we investigate how to use our QG model to augment QA datasets and enable semi-supervised QA. We propose two ways to generate synthetic QA pairs: generate new questions from existing articles or collect QA pairs from new articles. We also propose two empirically effective strategies, a data filter and mixing mini-batch training, to properly use the QG-generated data for QA. Experiments show that our method improves over both BiDAF and BERT QA baselines, even without introducing new articles.