Understanding narratives requires reading between the lines, which in turn, requires interpreting the likely causes and effects of events, even when they are not mentioned explicitly. In this paper, we introduce Cosmos QA, a large-scale dataset of 35,600 problems that require commonsense-based reading comprehension, formulated as multiple-choice questions. In stark contrast to most existing reading comprehension datasets where the questions focus on factual and literal understanding of the context paragraph, our dataset focuses on reading between the lines over a diverse collection of people's everyday narratives, asking such questions as "what might be the possible reason of ...?", or "what would have happened if ..." that require reasoning beyond the exact text spans in the context. To establish baseline performances on Cosmos QA, we experiment with several state-of-the-art neural architectures for reading comprehension, and also propose a new architecture that improves over the competitive baselines. Experimental results demonstrate a significant gap between machine (68.4\%) and human performance (94\%), pointing to avenues for future research on commonsense machine comprehension. Dataset, code and leaderboard is publicly available at https://wilburone.github.io/cosmos.