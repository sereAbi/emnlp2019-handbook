The task of bilingual dictionary induction (BDI) is commonly used for intrinsic evaluation of cross-lingual word embeddings. The largest dataset for BDI was generated automatically, so its quality is dubious. 
We study the composition and quality of the test sets for five diverse languages from this dataset, with concerning findings: (1) a quarter of the data consists of proper nouns, which can be hardly indicative of BDI performance, and (2) there are pervasive gaps in the gold-standard targets. 
These issues appear to affect the ranking between cross-lingual embedding systems on individual languages, and the overall degree to which the systems differ in performance. With proper nouns removed from the data, the margin between the top two systems included in the study grows from 3.4\% to 17.2\%. Manual verification of the predictions, on the other hand, reveals that gaps in the gold standard targets artificially inflate the margin between the two systems on English to Bulgarian BDI from 0.1\% to 6.7\%. 
We thus suggest that future research either avoids drawing conclusions from quantitative results on this BDI dataset, or accompanies such evaluation with rigorous error analysis.