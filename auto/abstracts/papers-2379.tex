Domain-specific training typically makes NLP systems work better. We show that this extends to cognitive modeling as well by relating the states of a neural phrase-structure parser to electrophysiological measures from human participants. These measures were recorded as participants listened to a spoken recitation of the same literary text that was supplied as input to the neural parser. Given more training data, the system derives a better cognitive model --- but only when the training examples come from the same textual genre. This finding is consistent with the idea that humans adapt syntactic expectations to particular genres during language comprehension (Kaan and Chun, 2018; Branigan and Pickering, 2017).