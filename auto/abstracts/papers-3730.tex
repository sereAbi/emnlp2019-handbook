Socio-economic conditions are difficult to measure. For example, the U.S. Bureau of Labor Statistics needs to conduct large-scale household surveys regularly to track the unemployment rate, an indicator widely used by economists and policymakers. We argue that events reported in streaming news can be used as "micro-sensors" for measuring socio-economic conditions. Similar to collecting surveys and then counting answers, it is possible to measure a socio-economic indicator by counting related events. In this paper, we propose Event-Centric Indicator Measure (ECIM), a novel approach to measure socio-economic indicators with events. We empirically demonstrate strong correlation between ECIM values to several representative indicators in socio-economic research.