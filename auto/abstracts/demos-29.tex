MY-AKKHARA is a method used to input Burmese texts encoded in the Unicode standard, based on commonly accepted Latin transcription. By using this method, arbitrary Burmese strings can be accurately inputted with 26 lowercase Latin letters. Meanwhile, the 26 uppercase Latin letters are designed as shortcuts of lowercase letter sequences. The frequency of Burmese characters is considered in MY-AKKHARA to realize an efficient keystroke distribution on a QWERTY keyboard. Given that the Unicode standard has not been extensively used in digitization of Burmese, we hope that MY-AKKHARA can contribute to the widespread use of Unicode in Myanmar and can provide a platform for smart input methods for Burmese in the future. An implementation of MY-AKKHARA running in Windows is released at http://www2.nict.go.jp/astrec-att/member/ding/my-akkhara.html