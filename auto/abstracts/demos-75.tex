For languages with simple morphology such as English, automatic annotation pipelines such as spaCy or Stanford's CoreNLP successfully serve projects in academia and the industry. 
For many morphologically-rich languages (MRLs), similar pipelines show sub-optimal performance that limits their applicability for text analysis in research and the industry. 
The sub-optimal performance is mainly due to errors in early morphological disambiguation decisions, that cannot be recovered later on in the pipeline, yielding incoherent  annotations on the whole.  This paper describes the design and use of the ONLP suite,  a joint morpho-syntactic infrastructure for processing Modern Hebrew texts. The  joint inference over morphology and syntax substantially limits error propagation, and leads to high accuracy.  ONLP provides rich and expressive annotations which already serve diverse academic  and commercial needs. Its accompanying demo further serves educational activities, introducing Hebrew NLP intricacies to researchers and non-researchers alike.