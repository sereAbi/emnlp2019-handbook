In argumentation, framing is used to emphasize a specific aspect of a controversial topic while concealing others. When talking about legalizing drugs, for instance, its economical aspect may be emphasized. In general, we call a set of arguments that focus on the same aspect a frame. An argumentative text has to serve the ``right'' frame(s) to convince the audience to adopt the author's stance (e.g., being pro or con legalizing drugs). More specifically, an author has to choose frames that fit the audience's cultural background and interests.
This paper introduces frame identification, which is the task of splitting a set of arguments into non-overlapping frames. We present a fully unsupervised approach to this task, which first removes topical information and then identifies frames using clustering. For evaluation purposes, we provide a corpus with 12, 326 debate-portal arguments, organized along the frames of the debates' topics. On this corpus, our approach outperforms different strong baselines, achieving an F1-score of 0.28.