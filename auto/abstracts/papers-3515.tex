There is an extensive history of scholarship into what constitutes a ``basic'' color term, as well as a broadly attested acquisition sequence of basic color terms across many languages, as articulated in the seminal work of Berlin and Kay (1969). This paper employs a set of diverse measures on massively cross-linguistic data to operationalize and critique the Berlin and Kay color term hypotheses. Collectively, the 14 empirically-grounded computational linguistic metrics we design - as well as their aggregation - correlate strongly with both the Berlin and Kay basic/secondary color term partition ($\gamma$ = 0.96) and their hypothesized universal acquisition sequence. The measures and result provide further empirical evidence from computational linguistics in support of their claims, as well as additional nuance: they suggest treating the partition as a spectrum instead of a dichotomy.