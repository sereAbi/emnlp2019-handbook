Stylistic variation in text needs to be studied with different aspects including the writer’s personal traits, interpersonal relations, rhetoric, and more. Despite recent attempts on computational modeling of the variation, the lack of parallel corpora of style language makes it difficult to systematically control the stylistic change as well as evaluate such models. We release PASTEL, the parallel and annotated stylistic language dataset, that contains {\textasciitilde}41K parallel sentences (8.3K parallel stories) annotated across different personas. Each persona has different styles in conjunction: gender, age, country, political view, education, ethnic, and time-of-writing. The dataset is collected from human annotators with solid control of input denotation: not only preserving original meaning between text, but promoting stylistic diversity to annotators. We test the dataset on two interesting applications of style language, where PASTEL helps design appropriate experiment and evaluation. First, in predicting a target style (e.g., male or female in gender) given a text, multiple styles of PASTEL make other external style variables controlled (or fixed), which is a more accurate experimental design. Second, a simple supervised model with our parallel text outperforms the unsupervised models using nonparallel text in style transfer. Our dataset is publicly available.