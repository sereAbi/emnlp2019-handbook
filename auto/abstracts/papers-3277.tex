Research in natural language processing proceeds, in part, by demonstrating that new models achieve superior performance (e.g., accuracy) on held-out test data,  compared to previous  results. In this paper, we demonstrate that test-set performance scores alone are insufficient for  drawing accurate conclusions about which model performs best. We argue for reporting additional details, especially performance  on  validation  data  obtained  during model development.  We present a novel technique  for  doing  so: expected  validation  performance of the best-found model as a function of computation budget (i.e., the  number of hyperparameter search trials or the overall training  time). Using our approach, we find multiple recent model comparisons where authors would have reached a different conclusion if they had  used more (or less) computation. Our approach also allows us to estimate  the  amount  of  computation required to obtain a given accuracy; applying it to several recently published results yields massive variation across papers, from hours to weeks. We conclude with a set of best  practices for reporting experimental results which allow for robust future comparisons, and provide code to allow researchers to use our technique.