Dialogue systems benefit greatly from optimizing on detailed annotations, such as transcribed utterances, internal dialogue state representations and dialogue act labels. However, collecting these annotations is expensive and time-consuming, holding back development in the area of dialogue modelling.
In this paper, we investigate semi-supervised learning methods that are able to reduce the amount of required intermediate labelling. We find that by leveraging un-annotated data instead, the amount of turn-level annotations of dialogue state can be significantly reduced when building a neural dialogue system. Our analysis on the MultiWOZ corpus, covering a range of domains and topics, finds that annotations can be reduced by up to 30\% while maintaining equivalent system performance. We also describe and evaluate the first end-to-end dialogue model created for the MultiWOZ corpus.