Question generation (QG) is the task of generating a question from a reference sentence and a specified answer within the sentence. A major challenge in QG is to identify answer-relevant context words to finish the declarative-to-interrogative sentence transformation.
Existing sequence-to-sequence neural models achieve this goal by proximity-based answer position encoding under the intuition that neighboring words of answers are of high possibility to be answer-relevant. However, such intuition may not apply to all cases especially for sentences with complex answer-relevant relations. Consequently, the performance of these models drops sharply when the relative distance between the answer fragment and other non-stop sentence words that also appear in the ground truth question increases. To address this issue, we propose a method to jointly model the unstructured sentence and the structured answer-relevant relation (extracted from the sentence in advance) for question generation. Specifically, the structured answer-relevant relation acts as the to the point context and it thus naturally helps keep the generated question to the point, while the unstructured sentence provides the full information. Extensive experiments show that to the point context helps our question generation model achieve significant improvements on several automatic evaluation metrics. Furthermore, our model is capable of generating diverse questions for a sentence which conveys multiple relations of its answer fragment.