We present our initial evaluation of a prototype system designed to assist nurses in assigning subject headings to nursing narratives - written in the context of documenting patient care in hospitals. Currently nurses may need to memorize several hundred subject headings from standardized nursing terminologies when structuring and assigning the right section/subject headings to their text. Our aim is to allow nurses to write in a narrative manner without having to plan and structure the text with respect to sections and subject headings, instead the system should assist with the assignment of subject headings and restructuring afterwards. We hypothesize that this could reduce the time and effort needed for nursing documentation in hospitals. A central component of the system is a text classification model based on a long short-term memory (LSTM) recurrent neural network architecture, trained on a large data set of nursing notes. A simple Web-based interface has been implemented for user interaction. To evaluate the system, three nurses write a set of artificial nursing shift notes in a fully unstructured narrative manner, without planning for or consider the use of sections and subject headings. These are then fed to the system which assigns subject headings to each sentence and then groups them into paragraphs. Manual evaluation is conducted by a group of nurses. The results show that about 70\% of the sentences are assigned to correct subject headings. The nurses believe that such a system can be of great help in making nursing documentation in hospitals easier and less time consuming. Finally, various measures and approaches for improving the system are discussed.
