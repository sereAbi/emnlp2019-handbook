Existing recurrent neural language models often fail to capture higher-level structure present in text: for example, rhyming patterns present in poetry. Much prior work on poetry generation uses manually defined constraints which are satisfied during decoding using either specialized decoding procedures or rejection sampling. The rhyming constraints themselves are typically not learned by the generator. We propose an alternate approach that uses a structured discriminator to learn a poetry generator that directly captures rhyming constraints in a generative adversarial setup. By causing the discriminator to compare poems based only on a learned similarity matrix of pairs of line ending words, the proposed approach is able to successfully learn rhyming patterns in two different English poetry datasets (Sonnet and Limerick) without explicitly being provided with any phonetic information