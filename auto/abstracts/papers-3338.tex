Fact verification requires validating a claim in the context of evidence. We show, however, that in the popular FEVER dataset this might not necessarily be the case. Claim-only classifiers perform competitively with top evidence-aware models. In this paper, we investigate the cause of this phenomenon, identifying strong cues for predicting labels solely based on the claim, without considering any evidence. We create an evaluation set that avoids those idiosyncrasies. The performance of FEVER-trained models significantly drops when evaluated on this test set. Therefore, we introduce a regularization method which alleviates the effect of bias in the training data, obtaining improvements on the newly created test set. This work is a step towards a more sound evaluation of reasoning capabilities in fact verification models.