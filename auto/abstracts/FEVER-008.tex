This article  presents   the SIRIUS-LTG system  for  the  Fact Extraction and VERification (FEVER) SharedTask. It  consists  of  three  components:   1)Wikipedia  Page  Retrieval:   First  we  extract the entities in the claim,  then we find potential Wikipedia URI candidates for each of the entities  using  a  SPARQL  query  over  DBpedia 2)Sentence selection: We investigate various techniques i.e. Smooth Inverse Frequency(SIF), Word Mover's Distance (WMD), Soft-Cosine Similarity, Cosine similarity with uni-gram Term Frequency Inverse Document Frequency  (TF-IDF)  to  rank  sentences  by  their similarity to the claim.  3)Textual Entailment: We compare three models for the task of claim classification.  We apply a Decomposable Attention  (DA)  model  (Parikh  et  al.,  2016),  a Decomposed Graph Entailment (DGE) model (Khot et al., 2018) and a Gradient-Boosted Decision  Trees  (TalosTree)  model  (Sean  et  al.,2017)  for  this  task.    The  experiments  show that the pipeline with simple Cosine Similarity  using  TFIDF  in  sentence  selection  along with  DA  model  as  labelling model  achieves the  best  results  on  the  development  set  (F1evidence:   32.17,  label  accuracy:   59.61  andFEVER  score:  0.3778).   Furthermore,  it  obtains 30.19, 48.87 and 36.55 in terms of F1 evidence, label accuracy and FEVER score, respectively, on the test set. Our system ranks 15th among 23 participants in the shared task prior to any human-evaluation of the evidence.
