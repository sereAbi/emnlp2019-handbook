For psychiatric disorders such as schizophrenia, longer durations of untreated psychosis are associated with worse intervention outcomes. Data included in electronic health records (EHRs) can be useful for retrospective clinical studies, but much of this is stored as unstructured text which cannot be directly used in computation. Natural Language Processing (NLP) methods can be used to extract this data, in order to identify symptoms and treatments from mental health records, and temporally anchor the first emergence of these. We are developing an EHR corpus annotated with time expressions, clinical entities and their relations, to be used for NLP development. In this study, we focus on the first step, identifying time expressions in EHRs for patients with schizophrenia. We developed a gold standard corpus, compared this corpus to other related corpora in terms of content and time expression prevalence, and adapted two NLP systems for extracting time expressions. To the best of our knowledge, this is the first resource annotated for temporal entities in the mental health domain.
