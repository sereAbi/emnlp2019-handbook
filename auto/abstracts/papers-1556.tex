Spoken Language Understanding (SLU) mainly involves two tasks, intent detection and slot filling, which are generally modeled jointly in existing works. However, most existing models fail to fully utilize cooccurrence relations between slots and intents, which restricts their potential performance. To address this issue, in this paper we propose a novel Collaborative Memory Network (CM-Net) based on the well-designed block, named CM-block. The CM-block firstly captures slot-specific and intent-specific features from memories in a collaborative manner, and then uses these enriched features to enhance local context representations, based on which the sequential information flow leads to more specific (slot and intent) global utterance representations. Through stacking multiple CM-blocks, our CM-Net is able to alternately perform information exchange among specific memories, local contexts and the global utterance, and thus incrementally enriches each other. We evaluate the CM-Net on two standard benchmarks (ATIS and SNIPS) and a self-collected corpus (CAIS). Experimental results show that the CM-Net achieves the state-of-the-art results on the ATIS and SNIPS in most of criteria, and significantly outperforms the baseline models on the CAIS. Additionally, we make the CAIS dataset publicly available for the research community.