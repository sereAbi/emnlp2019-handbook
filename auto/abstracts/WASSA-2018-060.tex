Analysis of the topics mentioned and opinions expressed in parliamentary debate motions---or proposals--is difficult for human readers, but necessary for understanding and automatic processing of the content of the subsequent speeches. We present a dataset of debate motions with pre-existing 'policy' labels, and investigate the utility of these labels for simultaneous topic and opinion polarity analysis. For topic detection, we apply one-versus-the-rest supervised topic classification, finding that good performance is achieved in predicting the policy topics, and that textual features derived from the debate titles associated with the motions are particularly indicative of motion topic. We then examine whether the output could also be used to determine the positions taken by proposers towards the different policies by investigating how well humans agree in interpreting the opinion polarities of the motions. Finding very high levels of agreement, we conclude that the policies used can be reliable labels for use in these tasks, and that successful topic detection can therefore provide opinion analysis of the motions 'for free'.
