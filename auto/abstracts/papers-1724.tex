Game-theoretic models, thanks to their intrinsic ability to exploit contextual information, have shown to be particularly suited for the Word Sense Disambiguation task. They represent ambiguous words as the players of a non cooperative game and their senses as the strategies that the players can select in order to play the games. The interaction among the players is modeled with a weighted graph and the payoff as an embedding similarity function, that the players try to maximize. The impact of the word and sense embedding representations in the framework has been tested and analyzed extensively: experiments on standard benchmarks show state-of-art performances and different tests hint at the usefulness of using disambiguation to obtain contextualized word representations.