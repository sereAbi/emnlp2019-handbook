How much does ``free shipping!'' help an advertisement's ability to persuade? This paper presents two methods for performance attribution: finding the degree to which an outcome can be attributed to parts of a text while controlling for potential confounders. Both algorithms are based on interpreting the behaviors and parameters of trained neural networks. One method uses a CNN to encode the text, an adversarial objective function to control for confounders, and projects its weights onto its activations to interpret the importance of each phrase towards each output class. The other method leverages residualization to control for confounds and performs interpretation by aggregating over learned word vectors. We demonstrate these algorithms' efficacy on 118,000 internet search advertisements and outcomes, finding language indicative of high and low click through rate (CTR) regardless of who the ad is by or what it is for. Our results suggest the proposed algorithms are high performance and data efficient, able to glean actionable insights from fewer than 10,000 data points. We find that quick, easy, and authoritative language is associated with success, while lackluster embellishment is related to failure. These findings agree with the advertising industry's emperical wisdom, automatically revealing insights which previously required manual A/B testing to discover.
