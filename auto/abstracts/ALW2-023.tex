The paper investigates the potential effects user features have on hate speech classification. A quantitative analysis of Twitter data was conducted to better understand user characteristics, but no correlations were found between hateful text and the characteristics of the users who had posted it. However, experiments with a hate speech classifier based on datasets from three different languages showed that combining certain user features with textual features gave slight improvements of classification performance. While the incorporation of user features resulted in varying impact on performance for the different datasets used, user network-related features provided the most consistent improvements.
