Although WhatsApp is used by teenagers as one major channel of cyberbullying, such interactions remain invisible due to the app privacy policies that do not allow ex-post data  collection. Indeed, most of the information on these phenomena rely on surveys regarding self-reported data. In order to overcome this limitation, we describe in this paper the activities that led to the creation of a WhatsApp dataset to study cyberbullying among Italian students aged 12-13. We present not only the collected chats with annotations about user role and type of offense, but also the living lab created in a collaboration between researchers and schools to monitor and analyse cyberbullying. Finally, we discuss some open issues, dealing with ethical, operational and epistemic aspects.
