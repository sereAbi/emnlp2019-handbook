Despite detection of suicidal ideation on social media has made great progress in recent years, people's implicitly and anti-real contrarily expressed posts still remain as an obstacle,
constraining the detectors to acquire higher satisfactory performance. Enlightened by the hidden ``tree holes" phenomenon on microblog, where people at suicide risk tend to disclose their inner real feelings and thoughts to the microblog space whose authors have committed suicide, we explore the use of tree holes to enhance microblog-based suicide risk detection from the following two perspectives.
(1) We build suicide-oriented word embeddings based on tree hole contents
to strength the sensibility of suicide-related lexicons and context based on tree hole contents.
(2) A two-layered attention mechanism is deployed to grasp intermittently changing points
from individual's open blog streams, revealing one's inner emotional world more or less.
Our experimental results show that with suicide-oriented word embeddings and attention,
microblog-based suicide risk detection can achieve over 91\% accuracy.
A large-scale well-labelled suicide data set is also reported in the paper.