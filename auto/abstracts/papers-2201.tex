Emergent multi-agent communication protocols are very different from natural language and not easily interpretable by humans. We find that agents that were initially pretrained to produce natural language can also experience detrimental language drift: when a non-linguistic reward is used in a goal-based task, e.g. some scalar success metric, the communication protocol may easily and radically diverge from natural language. We recast translation as a multi-agent communication game and examine auxiliary training constraints for their effectiveness in mitigating language drift. We show that a combination of syntactic (language model likelihood) and semantic (visual grounding) constraints gives the best communication performance, allowing pre-trained agents to retain English syntax while learning to accurately convey the intended meaning.