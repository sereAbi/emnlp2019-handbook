According to screenwriting theory, turning points (e.g., change of plans, major setback, climax) are crucial narrative moments within a screenplay: they define the plot structure, determine its progression and segment the screenplay into thematic units (e.g., setup, complications, aftermath).  We propose the task of turning point identification in movies as a means of analyzing their narrative structure. We argue that turning points and the segmentation they provide can facilitate processing long, complex narratives, such as screenplays, for summarization and question answering. We introduce a dataset consisting of screenplays and plot synopses annotated with turning points and present an end-to-end neural network model that identifies turning points in plot synopses and projects them onto scenes in screenplays. Our model outperforms strong baselines based on state-of-the-art sentence representations and the expected position of turning points.