Condescending language use is caustic; it can bring dialogues to an end and bifurcate communities. Thus, systems for condescension detection could have a large positive impact. A challenge here is that condescension is often impossible to detect from isolated utterances, as it depends on the discourse and social context. To address this, we present TalkDown, a new labeled dataset of condescending linguistic acts in context. We show that extending a language-only model with representations of the discourse improves performance, and we motivate techniques for dealing with the low rates of condescension overall. We also use our model to estimate condescension rates in various online communities and relate these differences to differing community norms.