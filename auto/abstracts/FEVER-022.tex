Fact-checking is a journalistic practice that compares a claim made publicly against trusted sources of facts. Wang (2017) introduced a large dataset of validated claims from the POLITIFACT .com website (LIAR dataset), enabling the development of machine learning approaches for fact-checking. However, approaches based on this dataset have focused primarily on modeling the claim and speaker-related metadata, without considering the evidence used by humans in labeling the claims. We extend the LIAR dataset by automatically extracting the justification from the fact-checking article used by humans to label a given claim. We show that modeling the extracted justification in conjunction with the claim (and metadata) provides a significant improvement regardless of the machine learning model used (feature-based or deep learning) both in a binary classification task (true, false) and in a six-way classification task (pants on fire, false, mostly false, half true, mostly true, true).
