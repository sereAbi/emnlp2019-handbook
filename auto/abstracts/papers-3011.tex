Social media provides a timely yet challenging data source for adverse drug reaction (ADR) detection. Existing dictionary-based, semi-supervised learning approaches are intrinsically limited by the coverage and maintainability of laymen health vocabularies. In this paper, we introduce a data augmentation approach that leverages variational autoencoders to learn high-quality data distributions from a large unlabeled dataset, and subsequently, to automatically generate a large labeled training set from a small set of labeled samples. This allows for efficient social-media ADR detection with low training and re-training costs to adapt to the changes and emergence of informal medical laymen terms. An extensive evaluation performed on Twitter and Reddit data shows that our approach matches the performance of fully-supervised approaches while requiring only 25\% of training data.