This paper reports on the results of an empirical study of adjudicatory decisions about veterans' claims for disability benefits in the United States. It develops a typology of kinds of relevant evidence (argument premises) employed in cases, and it identifies factors that the tribunal considers when assessing the credibility or trustworthiness of individual items of evidence. It also reports on patterns or ``soft rules'' that the tribunal uses to comparatively weigh the probative value of conflicting evidence. These evidence types, credibility factors, and comparison patterns are developed to be inter-operable with legal rules governing the evidence assessment process in the U.S. This approach should be transferable to other legal and non-legal domains.
