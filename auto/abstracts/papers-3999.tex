We propose a deep factorization model for typographic analysis that disentangles content from style. Specifically, a variational inference procedure factors each training glyph into the combination of a character-specific content embedding and a latent font-specific style variable. The underlying generative model combines these factors through an asymmetric transpose convolutional process to generate the image of the glyph itself. When trained on corpora of fonts, our model learns a manifold over font styles that can be used to analyze or reconstruct new, unseen fonts. On the task of reconstructing missing glyphs from an unknown font given only a small number of observations, our model outperforms both a strong nearest neighbors baseline and a state-of-the-art discriminative model from prior work.