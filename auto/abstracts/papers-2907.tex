Monitoring patients in ICU is a challenging and high-cost task. Hence, predicting the condition of patients during their ICU stay can help provide better acute care and plan the hospital's resources. There has been continuous progress in machine learning research for ICU management, and most of this work has focused on using time series signals recorded by ICU instruments. In our work, we show that adding clinical notes as another modality improves the performance of the model for three benchmark tasks: in-hospital mortality prediction, modeling decompensation, and length of stay forecasting that play an important role in ICU management. While the time-series data is measured at regular intervals, doctor notes are charted at irregular times, making it challenging to model them together. We propose a method to model them jointly, achieving considerable improvement across benchmark tasks over baseline time-series model.