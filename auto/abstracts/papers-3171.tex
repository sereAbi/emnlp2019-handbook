Dependency parsing of conversational input can play an important role in language understanding for dialog systems by identifying the relationships between entities extracted from user utterances. Additionally, effective dependency parsing can elucidate differences in language structure and usage for discourse analysis of human-human versus human-machine dialogs. However, models trained on datasets based on news articles and web data do not perform well on spoken human-machine dialog, and currently available annotation schemes do not adapt well to dialog data. Therefore, we propose the Spoken Conversation Universal Dependencies (SCUD) annotation scheme that extends the Universal Dependencies (UD) (Nivre et al., 2016) guidelines to spoken human-machine dialogs. We also provide ConvBank, a conversation dataset between humans and an open-domain conversational dialog system with SCUD annotation. Finally, to demonstrate the utility of the dataset, we train a dependency parser on the ConvBank dataset. We demonstrate that by pre-training a dependency parser on a set of larger public datasets and fine-tuning on ConvBank data, we achieved the best result, 85.05\% unlabeled and 77.82\% labeled attachment accuracy.