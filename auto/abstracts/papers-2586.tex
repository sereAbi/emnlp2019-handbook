Traditionally, most data-to-text applications have been designed using a modular pipeline architecture, in which non-linguistic input data is converted into natural language through several intermediate transformations. By contrast, recent neural models for data-to-text generation have been proposed as end-to-end approaches, where the non-linguistic input is rendered in natural language with much less explicit intermediate representations in between. This study introduces a systematic comparison between neural pipeline and end-to-end data-to-text approaches for the generation of text from RDF triples. Both architectures were implemented making use of the encoder-decoder Gated-Recurrent Units (GRU) and Transformer, two state-of-the art deep learning methods. Automatic and human evaluations together with a qualitative analysis suggest that having explicit intermediate steps in the generation process results in better texts than the ones generated by end-to-end approaches. Moreover, the pipeline models generalize better to unseen inputs. Data and code are  publicly available.