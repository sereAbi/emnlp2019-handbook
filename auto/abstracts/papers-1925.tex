The objective of non-parallel text style transfer, or controllable text generation, is to alter specific attributes (e.g. sentiment, mood, tense, politeness, etc) of a given text while preserving its remaining attributes and content. Generative adversarial network (GAN) is a popular model to ensure the transferred sentences are realistic and have the desired target styles. However, training GAN often suffers from mode collapse problem, which causes that the transferred text is little related to the original text. In this paper, we propose a new GAN model with a word-level conditional architecture and a two-phase training procedure. By using a style-related condition architecture before generating a word, our model is able to maintain style-unrelated words while changing the others. By separating the training procedure into reconstruction and transfer phases, our model is able to learn a proper text generation process, which further improves the content preservation. We test our model on polarity sentiment transfer and multiple-attribute transfer tasks. The empirical results show that our model achieves comparable evaluation scores in both transfer accuracy and fluency but significantly outperforms other state-of-the-art models in content compatibility on three real-world datasets.