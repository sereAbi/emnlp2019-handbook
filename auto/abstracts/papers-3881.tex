We present CoSQL, a corpus for building cross-domain, general-purpose database (DB) querying dialogue systems. It consists of 30k+ turns plus 10k+ annotated SQL queries, obtained from a Wizard-of-Oz (WOZ) collection of 3k dialogues querying 200 complex DBs spanning 138 domains. Each dialogue simulates a real-world DB query scenario with a crowd worker as a user exploring the DB and a SQL expert retrieving answers with SQL, clarifying ambiguous questions, or otherwise informing of unanswerable questions. When
user questions are answerable by SQL, the expert describes the SQL and execution results to the user, hence maintaining a natural interaction flow. CoSQL introduces new challenges compared to existing task-oriented dialogue datasets: (1) the dialogue states are grounded
in SQL, a domain-independent executable representation, instead of domain-specific slot value pairs, and (2) because testing is done on unseen databases, success requires generalizing to new domains. CoSQL includes three tasks: SQL-grounded dialogue state tracking, response generation from query results, and user dialogue act prediction. We evaluate a set of strong baselines for each task and show that CoSQL presents significant challenges for future research. The dataset, baselines, and leaderboard will be released at https://yale-lily.github.io/cosql.