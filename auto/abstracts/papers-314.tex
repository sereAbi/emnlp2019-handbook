As an essential component of natural language processing, text classification relies on deep learning in recent years.
Various neural networks are designed for text classification on the basis of word embedding.
However, polysemy is a fundamental feature of the natural language, which brings challenges to text classification. 
One polysemic word contains more than one sense, while the word embedding procedure conflates different senses of a polysemic word into a single vector. 
Extracting the distinct representation for the specific sense could thus lead to fine-grained models with strong generalization ability.
It has been demonstrated that multiple senses of a word actually reside in linear superposition within the word embedding so that specific senses can be extracted from the original word embedding. 
Therefore, 
we propose to use capsule networks to construct the vectorized representation of semantics and utilize hyperplanes to decompose each capsule to acquire the specific senses.
A novel dynamic routing mechanism named `routing-on-hyperplane' will select the proper sense for the downstream classification task.
Our model is evaluated on 6 different datasets, and the experimental results show that our model is capable of extracting more discriminative semantic features and yields a significant performance gain compared to other baseline methods.