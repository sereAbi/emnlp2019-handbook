This paper provides a detailed comparison of a data programming approach with (i) off-the-shelf, state-of-the-art deep learning architectures that optimize their representations (BERT) and (ii) handcrafted-feature approaches previously used in the discourse analysis literature. We compare these approaches on the task of learning discourse structure for multi-party dialogue. The data programming paradigm offered by the Snorkel framework allows a user to label training data using expert-composed heuristics, which are then transformed via the ''generative step'' into probability distributions of the class labels given the data. We show that on our task the generative model outperforms both deep learning architectures as well as more traditional ML approaches when learning discourse structure---it even outperforms the combination of deep learning methods and hand-crafted features. We also implement several strategies for ''decoding'' our generative model output in order to improve our results. We conclude that weak supervision methods hold great promise as a means for creating and improving data sets for discourse structure.