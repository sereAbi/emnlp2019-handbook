For machine translation, a vast majority of language pairs in the world are considered
low-resource because they have little parallel data available. Besides the technical challenges of learning with limited supervision, it is difficult to evaluate methods trained on low-resource language pairs because of the lack of freely and publicly available benchmarks. In this work, we introduce the FLORES evaluation datasets for Nepali–English and Sinhala–
English, based on sentences translated from Wikipedia. Compared to English, these are languages with very different morphology and syntax, for which little out-of-domain parallel data is available and for which relatively large amounts of monolingual data are freely available. We describe our process to collect and cross-check the quality of translations,
and we report baseline performance using several learning settings: fully supervised, weakly supervised, semi-supervised, and fully unsupervised. Our experiments demonstrate that current state-of-the-art methods perform rather poorly on this benchmark, posing a challenge to the research community working on low-resource MT. Data and code to reproduce our
experiments are available at https://github.com/facebookresearch/flores.