Tracking entities in procedural language requires understanding the transformations arising from actions on entities as well as those entities' interactions. While self-attention-based pre-trained language encoders like GPT and BERT  have been successfully applied across a range of natural language understanding tasks, their ability to handle the nuances of procedural texts is still unknown.  In this paper, we explore the use of pre-trained transformer networks for entity tracking tasks in procedural text. First, we test standard lightweight approaches for prediction with pre-trained transformers, and find that these approaches underperforms even simple baselines. We show that much stronger results can be attained by restructuring the input to guide the model to focus on a particular entity. Second, we assess the degree to which the transformer networks capture the process dynamics, investigating such factors as merged entities and oblique entity references. On two different tasks, ingredient detection in recipes and QA over scientific processes, we achieve state-of-the-art results, but our models still largely attend to shallow context clues and do not form complex representations of intermediate process state.