Negation is a universal but complicated linguistic phenomenon, which has received considerable attention from the NLP community over the last decade, since a negated statement often carries both an explicit negative focus and implicit positive meanings. For the sake of understanding a negated statement, it is critical to precisely detect the negative focus in context. However, how to capture contextual information for negative focus detection is still an open challenge. To well address this, we come up with an attention-based neural network to model contextual information. In particular, we introduce a framework which consists of a Bidirectional Long Short-Term Memory (BiLSTM) neural network and a Conditional Random Fields (CRF) layer to effectively encode the order information and the long-range context dependency in a sentence. Moreover, we design two types of attention mechanisms, word-level contextual attention and topic-level contextual attention, to take advantage of contextual information across sentences from both the word perspective and the topic perspective, respectively. Experimental results on the SEM'12 shared task corpus show that our approach achieves the best performance on negative focus detection, yielding an absolute improvement of 2.11\% over the state-of-the-art. This demonstrates the great effectiveness of the two types of contextual attention mechanisms.