In Natural Language Understanding, the task of response generation is usually focused on responses to short texts, such as tweets or a turn in a dialog. Here we present a novel task of producing a critical response to a long argumentative text, and suggest a method based on general rebuttal arguments to address it. We do this in the context of the recently-suggested task of listening comprehension over argumentative content: given a speech on some specified topic, and a list of relevant arguments, the goal is to determine which of the arguments appear in the speech. The general rebuttals we describe here (in English) overcome the need for topic-specific arguments to be provided, by proving to be applicable for a large set of topics. This allows creating responses beyond the scope of topics for which specific arguments are available. All data collected during this work is freely available for research.