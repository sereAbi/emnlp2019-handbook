Sentiment and topic analysis are common methods used for social media monitoring. Essentially, these methods answers questions such as, ``What is being talked about, regarding X'', and ``What do people feel, regarding X''. In this paper, we investigate another venue for social media monitoring, namely issue ownership. In political science, issue ownership has been used to explain voter choice and electoral outcomes. The theory states that voters value certain issues, and cast votes according to the party which they feel best address these issues. We argue that issue alignment can be seen as a kind of semantic source similarity of the kind ``How similar is source A to issue owner P, when talking about issue X'', and as such can be measured using Word/Document embedding techniques. We present work in progress towards measuring that kind of conditioned similarity, and introduce a new notion of similarity for predictive embeddings. We then test this method by measuring the similarity between politically aligned media and political parties, conditioned on Bloc-specific issues.
