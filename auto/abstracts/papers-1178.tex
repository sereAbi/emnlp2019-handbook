A key challenge in topic-focused summarization is determining what information should be included in the summary, a problem known as content selection. In this work, we propose a new method for studying content selection in topic-focused summarization called the summary cloze task. The goal of the summary cloze task is to generate the next sentence of a summary conditioned on the beginning of the summary, a topic, and a reference document(s). The main challenge is deciding what information in the references is relevant to the topic and partial summary and should be included in the summary. Although the cloze task does not address all aspects of the traditional summarization problem, the more narrow scope of the task allows us to collect a large-scale datset of nearly 500k summary cloze instances from Wikipedia. We report experimental results on this new dataset using various extractive models and a two-step abstractive model that first extractively selects a small number of sentences and then abstractively summarizes them. Our results show that the topic and partial summary help the models identify relevant content, but the task remains a significant challenge.