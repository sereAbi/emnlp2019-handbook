We consider open-domain question answering (QA) where answers are drawn from either a corpus, a knowledge base (KB), or a combination of both of these. We focus on a setting in which a corpus is supplemented with a large but incomplete KB, and on questions that require non-trivial (e.g., multi-hop) reasoning. We describe PullNet, an integrated framework for (1) learning what to retrieve and (2) reasoning with this heterogeneous information to find the best answer. PullNet uses an iterative process to construct a question-specific subgraph that contains information relevant to the question. In each iteration, a graph convolutional network (graph CNN) is used to identify subgraph nodes that should be expanded using retrieval (orpull) operations on the corpus and/or KB. After the subgraph is complete, another graph CNN is used to extract the answer from the subgraph. This retrieve-and-reason process allows us to answer multi-hop questions using large KBs and corpora. PullNet is weakly supervised, requiring question-answer pairs but not gold inference paths. Experimentally PullNet improves over the prior state-of-the art, and in the setting where a corpus is used with incomplete KB these improvements are often dramatic. PullNet is also often superior to prior systems in a KB-only setting or a text-only setting.
