Data-driven statistical Natural Language Processing (NLP) techniques leverage large amounts of language data to build models that can understand language. However, most language data reflect the public discourse at the time the data was produced, and hence NLP models are susceptible to learning incidental associations around named referents at a particular point in time, in addition to general linguistic meaning. An NLP system designed to model notions such as sentiment and toxicity should ideally produce scores that are independent of the identity of such entities mentioned in text and their social associations. For example, in a general purpose sentiment analysis system, a phrase such as I hate Katy Perry should be interpreted as having the same sentiment as I hate Taylor Swift. Based on this idea, we propose a generic evaluation framework, Perturbation Sensitivity Analysis, which detects unintended model biases related to named entities, and requires no new annotations or corpora. We demonstrate the utility of this analysis by employing it on two different NLP models --- a sentiment model and a toxicity model --- applied on online comments in English language from four different genres.