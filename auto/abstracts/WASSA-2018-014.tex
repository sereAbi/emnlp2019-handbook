Negation is one of the shifters or operators that can change the semantic orientation of a word or a sentence and, consequently, it has to be taken into consideration in sentiment analysis. In this work, we have analyzed the effects of negation on the semantic orientation in Basque. The analysis shows that negation markers can strengthen, weaken or have no effect on sentiment orientation of a word or a group of words. Using the Constraint Grammar formalism, we have designed and evaluated a set of linguistic rules to formalize these three phenomena. The results show that two phenomena, strengthening and no change, have been identified accurately and the third one, weakening, with acceptable results.
