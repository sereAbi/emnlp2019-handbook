Contextualized word embeddings such as ELMo and BERT provide a foundation for strong performance across a wide range of natural language processing tasks by pretraining on large corpora of unlabeled text. However, the applicability of this approach is unknown when the target domain varies substantially from the pretraining corpus. We are specifically interested in the scenario in which labeled data is available in only a canonical source domain such as newstext, and the target domain is distinct from both the labeled and pretraining texts. To address this scenario, we propose domain-adaptive fine-tuning, in which the contextualized embeddings are adapted by masked language modeling on text from the target domain. We test this approach on sequence labeling in two challenging domains: Early Modern English and Twitter. Both domains differ substantially from existing pretraining corpora, and domain-adaptive fine-tuning yields substantial improvements over strong BERT baselines, with particularly impressive results on out-of-vocabulary words. We conclude that domain-adaptive fine-tuning offers a simple and effective approach for the unsupervised adaptation of sequence labeling to difficult new domains.