Generative classifiers offer potential advantages over their discriminative counterparts, namely in the areas of data efficiency, robustness to data shift and adversarial examples, and zero-shot learning (Ng and Jordan,2002; Yogatama et al., 2017; Lewis and Fan,2019). In this paper, we improve generative text classifiers by introducing discrete latent variables into the generative story, and explore several graphical model configurations. We parameterize the distributions using standard neural architectures used in conditional language modeling and perform learning by directly maximizing the log marginal likelihood via gradient-based optimization, which avoids the need to do expectation-maximization. We empirically characterize the performance of our models on six text classification datasets. The choice of where to include the latent variable has a significant impact on performance, with the strongest results obtained when using the latent variable as an auxiliary conditioning variable in the generation of the textual input. This model consistently outperforms both the generative and discriminative classifiers in small-data settings. We analyze our model by finding that the latent variable captures interpretable properties of the data, even with very small training sets.