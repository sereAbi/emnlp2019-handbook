Legal Tech is developed to help people with legal services and solve legal problems via machines. To achieve this, one of the key requirements for machines is to utilize legal knowledge and comprehend legal context. This can be fulfilled by natural language processing (NLP) techniques, for instance, text representation, text categorization, question answering (QA) and natural language inference, etc. To this end, we introduce a freely available Chinese Legal Tech system (IFlyLegal) that benefits from multiple NLP tasks. It is an integrated system that performs legal consulting, multi-way law searching, and legal document analysis by exploiting techniques such as deep contextual representations and various attention mechanisms. To our knowledge, IFlyLegal is the first Chinese legal system that employs up-to-date NLP techniques and caters for needs of different user groups, such as lawyers, judges, procurators, and clients. Since Jan, 2019, we have gathered 2,349 users and 28,238 page views (till June, 23, 2019).