A new law was established in Japan to promote utilization of EHRs for research and developments, while de-identification is required to use EHRs. However, studies of automatic anonymization in the healthcare domain is not active for Japanese language, no de-identification tool available in practical performance for Japanese medical domains, as far as we know. Previous works show that rule-based methods are still effective, while deep learning methods are reported to be better recently. In order to implement and evaluate an de-identification tool in a practical level, we implemented three methods, rule-based, CRF, and LSTM. We prepared three datasets of pseudo EHRs with de-identification tags manually annoated. These datasets are derived from shared task data to compare with previous works, and our new data to increase training data. Our result shows that our LSTM-based method is better and robust, which leads to our future work that plans to apply our system to actual de-identification tasks in hospitals.
