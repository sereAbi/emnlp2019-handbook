We present set to ordered text, a natural language generation task applied to automatically generating discharge instructions from admission ICD (International Classification of Diseases) codes. 
This task differs from other natural language generation tasks in the following ways: 
(1) The input is a set of identifiable entities (ICD codes) where the relations between individual entity are not explicitly specified. 
(2) The output text is not a narrative description (e.g. news articles) composed from the input. Rather, inferences are made from the input (symptoms specified in ICD codes) to generate the output (instructions). 
(3) There is an optimal order in which each sentence (instruction) should appear in the output. Unlike most other tasks, neither the input (ICD codes) nor their corresponding symptoms appear in the output, so the ordering of the output instructions needs to be learned in an unsupervised fashion. 
Based on clinical intuition, we hypothesize that each instruction in the output is mapped to a subset of ICD codes specified in the input.
We propose a neural architecture that jointly models (a) subset selection: choosing relevant subsets from a set of input entities; (b) content ordering: learning the order of instructions; and (c) text generation:  representing the instructions corresponding to the selected subsets in natural language. In addition, we penalize redundancy during beam search to improve tractability for long text generation. Our model outperforms baseline models in BLEU scores and human evaluation. We plan to extend this work to other tasks such as recipe generation from ingredients.