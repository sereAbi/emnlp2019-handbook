Traditional recommendation systems produce static rather than interactive recommendations invariant to a user’s specific requests, clarifications, or current mood, and can suffer from the cold-start problem if their tastes are unknown. These issues can be alleviated by treating recommendation as an interactive dialogue task instead, where an expert recommender can sequentially ask about someone’s preferences, react to their requests, and recommend more appropriate items. In this work, we collect a goal-driven recommendation dialogue dataset (GoRecDial), which consists of 9,125 dialogue games and 81,260 conversation turns between pairs of human workers recommending movies to each other. The task is specifically designed as a cooperative game between two players working towards a quantifiable common goal. We leverage the dataset to develop an end-to-end dialogue system that can simultaneously converse and recommend. Models are first trained to imitate the behavior of human players without considering the task goal itself (supervised training). We then finetune our models on simulated bot-bot conversations between two paired pre-trained models (bot-play), in order to achieve the dialogue goal. Our experiments show that models finetuned with bot-play learn improved dialogue strategies, reach the dialogue goal more often when paired with a human, and are rated as more consistent by humans compared to models trained without bot-play. The dataset and code are publicly available through the ParlAI framework.