\begin{bio}

\textbf{Lili Mou} is currently a postdoctoral fellow at the University of Waterloo. Lili Mou received
his BS and PhD degrees from School of EECS, Peking University. His research interests include
deep learning applied to natural language process ing, especially focusing on neural-symbolic approaches and generative models for NLP. He has publications at top-tier conferences and journals such as AAAI, ACL, CIKM, COLING, EMNLP, ICML, IJCAI, INTERSPEECH, and TACL, as well as a monograph with Springer. He has served as a PC member/reviewer at AAAI, ACL, COLING, IJCAI, NAACL, and TKDE.


\textbf{Hao Zhou}  is a research scientist at ByteDance AI Lab. His research interests are machine learning and its applications for natural language processing, including syntax parsing, machine translation and text generation. Currently, he focuses on deep generative models for NLP. He received his Ph.D. degree in computer science from Nanjing University. He has publications in prestigious conferences and journals, including ACL, EMNLP, NIPS, AAAI, TACL and JAIR. He has served top conferences as a PC member, including ACL, EMNLP, IJCAI, and AAAI.


\textbf{Lei Li} is the Director of ByteDance AI Lab. Lei received his B.S. in Computer Science and Engineering from Shanghai Jiao Tong University (ACM class) and Ph.D. in Computer Science from Carnegie Mellon University, respectively. His dissertation work on fast algorithms for mining co-evolving time series was awarded ACM KDD best dissertation (runner up). His recent work on AI writing received 2nd-class award of WU Wenjun AI prize. Before ByteDance, he worked at Baidu’s Institute of Deep Learning in Silicon Valley as a Principal Research Scientist. Before that, he was working in EECS department of UC Berkeley as a Post-Doctoral Researcher. He has served in the Program Committee for ICML 2014, ECML/PKDD 2014/2015, SDM 2013/2014, IJCAI 2011/2013/2016/2019, KDD 2015/2016, 2017 KDD Cup co-Chair, KDD 2018 hands-on tutorial co-chair, EMNLP 2018, AAAI 2019 senior PC, and as a lecturer in 2014 summer school on Probabilistic Programming for Advancing Machine Learning. He has published over 40 technical papers and holds 3 US patents.


\end{bio}

\begin{tutorial}
  {Discreteness in Neural Natural Language Processing}
  {tutorial-final-005}
  {\daydateyear, \tutorialmorningtime}
  {\TutLocE}

This tutorial provides a comprehensive guide to the process of discreteness in neural NLP.

As a gentle start, we will briefly introduce the background of deep learning based NLP, where we point out the ubiquitous discreteness of natural language and its challenges in neural information processing. Particularly, we will focus on how such discreteness plays a role in the input space, the latent space, and the output space of a neural network. In each part, we will provide examples, discuss machine learning techniques, as well as demonstrate NLP applications.

\end{tutorial}
