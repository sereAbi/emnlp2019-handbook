\begin{bio}

\textbf{Wei Wu} is an applied scientist in Microsoft XiaoIce team from 2018. Before that, he was a lead
researcher of Microsoft Research Asia (MSRA) from 2012 to 2017. He obtained a B.S. in Applied
Mathematics from Peking University in 2007 and earned Ph.D. in Applied Mathematics from Peking
University in 2012. His research interests include machine learning, natural language processing
(NLP), and information retrieval. His current research focus is building conversational engines for
chatbots with machine learning and NLP techniques. he has been working on single-turn
conversation and multi-turn conversation. He is a key technology contributor to core chat engines
of Microsoft XiaoIce and Microsoft Rinna. His recent achievement with XiaoIce team is launching
a fully generative chatbot in Indonesia with full dialogue generation technologies. The chatbot
now has more than 1.5 million users on LINE Indonesia.

\textbf{Rui Yan} is a tenure-track assistant professor at Peking University, and is dual affiliated
with Beijing Institute of Big Data Research. He is also an adjunct professor at central China Normal
University and Central University of Finance and Economics. Before he returned to academia, he was
a senior researcher at Baidu Inc. For the past 10 years, he has been working on Artificial Intelligence
(AI) for Natural Language Processing (NLP). Rui Yan has a broad interest in real world problems
related to natural languages, text information, social media, and web applications. Rui's research
focuses on Natural Language Processing, Information Retrieval, Machine Learning and Artificial
Intelligence. More specifically, he conducts research into conversational systems, natural
language cognition and generation, as well as NLP-related applications (i.e., summarization,
artistic writing, etc.). When he worked in Natural Language Processing Department (now AI Group)
at Baidu, his team launched the conversational system product from scratch, named Duer ChatBot.

\end{bio}

\begin{tutorial}
  {Deep Chit-Chat: Deep Learning for ChatBots}
  {tutorial-final-006}
  {\daydateyear, \tutorialafternoontime}
  {\TutLocF}

The tutorial is based on the long-term efforts on building conversational models with deep learning approaches for chatbots. We will summarize the fundamental challenges in modeling open domain dialogues, clarify the difference from modeling goal-oriented dialogues, and give an overview of state-of-the-art methods for open domain conversation including both retrieval-based methods and generation-based methods. In addition to these, our tutorial will also cover some new trends of research of chatbots, such as how to design a reasonable evaluation system and how to "control" conversations from a chatbot with some specific information such as personas, styles, and emotions, etc.

\end{tutorial}
