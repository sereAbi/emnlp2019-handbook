\begin{bio}
\textbf{Tsung-Hsien Wen} is a co-founder and CTO of PolyAI, a London-based company looking to use the latest developments in NLP to create a general machine learning platform for deploying spoken dialogue systems. He holds a PhD from the Dialogue Systems group, University of Cambridge, where he worked under the supervision of Professor Steve Young. His research focuses on language generation and end-to-end dialogue modeling, specifically in learning to generate responses for task-oriented dialogue systems. He received the best paper award at EMNLP 2015. He was the tutor of the ”Deep Learning and NLG” tutorial at INLG 2016 and has given invited seminars to research groups at Google, Apple, Xerox, and Baidu China. He was also the invited lecturer for Samsung’s corporate training course in Warsaw and a research consultant at IPSoft Amelia team. He is one of the organisers of the 1st Workshop on NLP for Conversational AI at ACL 2019.


\textbf{Pei-Hao Su} is a co-founder and Chief Scientist of PolyAI, a London-based company looking to use the latest developments in NLP to create a general machine learning platform for deploying spoken dialogue systems. He holds a PhD from the Dialogue Systems group, University of Cambridge, where he worked under the supervision of Professor Steve Young. His research interests centre on applying deep learning, reinforcement learning and Bayesian approaches to dialogue management and reward estimation, with the aim of building systems that can learn directly from human interaction. He co-lectured a tutorial on deep learning for conversational AI at NAACL 2018. He has given several invited talks at academia and industry such as Apple, Microsoft, General Motors and DeepHack.Turing. He received the best student paper award at ACL 2016. He is one of the organisers of the 1st Workshop on NLP for Conversational AI at ACL 2019.


\textbf{Pawel Budzianowski} is a PhD student under the supervision of Prof. Anna Korhonen and Dr. Richard Turner in Dialogue Systems Group at Cambridge University. His research interests include multi-domain dialogue policy management, data collection for end-to-end dialogue systems and applied Bayesian deep learning. He received the best student paper award at ICASSP 2018 and the best resource paper award at EMNLP 2018. He co-lectured a tutorial on conversational AI at the Polish View on Machine Learning 2018 and gave a number of invited talks at academia and industry such as Apple, Google, and Toshiba.


\textbf{I\~nigo Casanueva} is a Machine Learning engineer at PolyAI, a London-based company looking to use the latest developments in NLP to create a general machine learning platform for deploying spoken dialogue systems. He got his PhD from the University of Sheffield and later he worked as Research Assistant in the Dialogue Systems group, University of Cambridge. His main research interest focuses on machine learning based dialogue management, especially its scalability and evaluation. He has published several papers on the topic as the main author, two of them nominated to best paper award, and has co-lectured a tutorial on deep learning for conversational AI at NAACL 2019. 


\textbf{Ivan Vuli\'c} is a Senior Research Associate in the Language Technology Lab at the University of
Cambridge, and a Senior Scientist at PolyAI. He holds a PhD from KU Leuven, obtained summa
cum laude. Ivan is interested in representation learning, human language understanding, distributional, lexical, and multi-modal semantics in monolingual and multilingual contexts, and transfer learning for enabling cross-lingual NLP applications and bringing conversational AI to resource-poor languages. He has co-lectured a tutorial on monolingual and multilingual topic models and applications at ECIR 2013 and WSDM 2014, a tutorial on word vector space specialisation at EACL 2017 and ESSLLI 2018, a tutorial on cross-lingual word representations at EMNLP 2017, and several tutorials on conversational AI, including the ones at the Conversational Intelligence Summer School 2018 and NAACL 2018. He has given a large number of invited talks at academia and industry.


\end{bio}

\begin{tutorial}
  {Data Collection and End-to-End Learning for Conversational AI}
  {tutorial-final-003}
  {\daydateyear, \tutorialafternoontime}
  {\TutLocC}

A fundamental long-term goal of conversational AI is to merge two main dialogue system paradigms into a standalone multi-purpose system. Such a system should be capable of conversing about arbitrary topics (Paradigm 1: open-domain dialogue systems), and simultaneously assist humans with completing a wide range of tasks with well-defined semantics such as restaurant search and booking, customer service applications, or ticket bookings (Paradigm 2: task-based dialogue systems).

The recent developmental leaps in conversational AI technology are undoubtedly linked to more and more sophisticated deep learning algorithms that capture patterns in increasing amounts of data generated by various data collection mechanisms. The goal of this tutorial is therefore twofold. First, it aims at familiarising the research community with the recent advances in algorithmic design of statistical dialogue systems for both open-domain and task-based dialogue paradigms. The focus of the tutorial is on recently introduced end-to-end learning for dialogue systems and their relation to more common modular systems. In theory, learning end-to-end from data offers seamless and unprecedented portability of dialogue systems to a wide spectrum of tasks and languages. From a practical point of view, there are still plenty of research challenges and opportunities remaining: in this tutorial we analyse this gap between theory and practice, and introduce the research community with the main advantages as well as with key practical limitations of current end-to-end dialogue learning.

The critical requirement of each statistical dialogue system is the data at hand. The system cannot provide assistance for the task without having appropriate task-related data to learn from. Therefore, the second major goal of this tutorial is to provide a comprehensive overview of the current approaches to data collection for dialogue, and analyse the current gaps and challenges with diverse data collection protocols, as well as their relation to and current limitations of data-driven end-to-end dialogue modeling. We will again analyse this relation and limitations both from research and industry perspective, and provide key insights on the application of state-of-the-art methodology into industry-scale conversational AI systems.

\end{tutorial}
