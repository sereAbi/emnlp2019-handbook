



\begin{tutorial}
  {Bias and Fairness in Natural Language Processing}
  {tutorial-final-004}
  {\daydateyear, \tutorialmorningtime}
  {\TutLocD}

Recent advances in data-driven machine learning techniques (e.g., deep neural networks) have revolutionized many natural language processing applications. These approaches automatically learn how to make decisions based on the statistics and diagnostic information from large amounts of training data. Despite the remarkable accuracy of machine learning in various applications, learning algorithms run the risk of relying on societal biases encoded in the training data to make predictions. This often occurs even when gender and ethnicity information is not explicitly provided to the system because learning algorithms are able to discover implicit associations between individuals and their demographic information based on other variables such as names, titles, home addresses, etc. Therefore, machine learning algorithms risk potentially encouraging unfair and discriminatory decision making and raise serious privacy concerns. Without properly quantifying and reducing the reliance on such correlations, broad adoption of these models might have the undesirable effect of magnifying harmful stereotypes or implicit biases that rely on sensitive demographic attributes.

In this tutorial, we will review the history of bias and fairness studies in machine learning and language processing and present recent community effort in quantifying and mitigating bias in natural language processing models for a wide spectrum of tasks, including word embeddings, co-reference resolution, machine translation, and vision-and-language tasks. In particular, we will focus on the following topics:

\begin{itemize}
\item Definitions of fairness and bias.

\item Data, algorithms, and models that propagate and even amplify social bias to NLP applications and metrics to quantify these biases.

\item Algorithmic solutions; learning objective; design principles to prevent social bias in NLP systems and their potential drawbacks.
\end{itemize}

The tutorial will bring researchers and practitioners to be aware of this issue, and encourage the research community to propose innovative solutions to promote fairness in NLP.


\end{tutorial}

\clearpage
\begin{bio}

\textbf{Kai-Wei Chang} is an assistant professor in the Department of Computer Science at the University of California Los Angeles. His research interests include designing robust machine learning methods for large and complex data and building language processing models for social good applications. His awards include the EMNLP Best Long Paper Award (2017), the KDD Best Paper Award (2010), and the Okawa Research Grant Award (2018). Kai-Wei has given tutorials at NAACL 15, AAAI 16 on different research topics, and gave a tutorial about gender stereotypes in word embeddings at FAT 18. 


\textbf{Margaret Mitchell} is a Senior Research Scientist and leads the Ethical AI team within Google Research. Her research is interdisciplinary, combining computer vision, natural language processing, statistical methods, deep learning, and cognitive science; and she applies her work in clinical and assistive domains. She has published over 40 papers, including top-tier conferences for NLP, Computer Vision, and Cognitive Science. She is also the co-founder of the annual workshops Clinical Psychology and Computational Linguistics, Ethics in Natural Language Processing, and Women and Underrepresented Minorities in Natural Language Processing. Her TED talk on evolving Artificial Intelligence towards positive goals has over one million views, and the system she co-developed using her first-place image-captioning system, Seeing AI, has won the Helen Keller Achievement Award award and the Fast Company Innovation by Design award.


\textbf{Vicente Ordonez} is an assistant professor in the Department of Computer Science at the University of Virginia. His research interests lie at the intersection of computer vision, natural language processing and machine learning. His focus is in building efficient visual recognition models that can perform high-level perceptual tasks for applications in social media, urban computing, and everyday activities that leverage both images and text. He is a recipient of best paper awards at the Conference on Empirical Methods in Natural Language Processing (2017) and the International Conference on Computer Vision (2013), an IBM Faculty Award (2017) and a Google Faculty Research Award (2017).


\end{bio}
