%\thispagestyle{myheadings}
\section{Invited Talk: Johan Bos}
\index{Bos, Johan}

\begin{center}
\begin{Large}
    {\bfseries\Large The Moment of Meaning and the Future of Computational Semantics} 
\vspace{1em}\par
\end{Large}

\daydateyear, 15:00--16:00 \vspace{1em}\\
%\PlenaryLoc \\
\vspace{1em}\par
%\includegraphics[height=100px]{content/tuesday/popovic-headshot.jpg}
\end{center}

\noindent
{\bfseries Abstract:} There are many recent advances in semantic parsing: we see a rising number of semantically annotated corpora and there is exciting technology (such as neural networks) to be explored. In this talk I will discuss what role computational semantics could play in future natural language processing applications (including fact checking and machine translation). I will argue that we should not just look at semantic parsing, but that things can get really interesting when we can use language-neutral meaning representations to draw (transparent) inferences. The main ideas will be exemplified by the parallel meaning bank, a new corpus comprising texts annotated with formal meaning representations for English, Dutch, German and Italian.

\vspace{3em}\par 

\vfill
\noindent

{\bfseries Biography:} 
Johan Bos is Professor of Computational Semantics at the University of Groningen (Netherlands). He received his doctorate from the Computational Linguistics Department at the University of the Saarland (Germany) and held post-doc positions at the University of Edinburgh (UK) and the La Sapienza University in Rome (Italy). In 2010, he moved to his current position in Groningen, leading the computational semantics group. Bos is the developer of Boxer, a state-of-the-art wide-coverage semantic parser for English, initiator of the Groningen Meaning Bank, a large semantically-annotated corpus of texts, and inventor of Wordrobe, a game with a purpose for semantic annotation. Bos received a \$1.5-million Vici grant from NWO (Netherlands Organisation for Scientific Research) in 2015 to investigate the role of meaning in human and machine translation. A concrete outcome of this project is the Parallel Meaning Bank containing detailed meaning representations for English, German, Dutch and Italian sentences.

\newpage
