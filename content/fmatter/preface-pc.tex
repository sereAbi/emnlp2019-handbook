\section{Message from the Program Committee Co-Chairs}
\setheaders%
    {Message from the Program Committee Co-Chairs}%
    {Message from the Program Committee Co-Chairs}
\thispagestyle{emptyheader}


\setlength{\parskip}{.7ex}

Welcome to EMNLP-IJCNLP 2019, the first joint EMNLP and IJCNLP conference! While this is the first time IJCNLP is held in Hong Kong, EMNLP left its footprints here in 2000 when it was still a tiny conference. There is nothing more exciting than seeing it return to Hong Kong after 19 years as one of the largest NLP conferences.

Owing to a significant increase in the number of submissions to recent NLP conferences, we, for the first time in the history of EMNLP and IJCNLP, attempted to reduce workload for reviewers by implementing a dual submission policy, where we disallowed authors to submit papers that are under review by a journal or another conference at the time of submission. Despite this policy, EMNLP-IJCNLP received 2914 submissions (excluding those withdrawn by the authors after initial submissions). This is an increase of about 30\% compared with EMNLP 2018, making EMNLP-IJCNLP 2019 the largest NLP conference ever! Out of the 2914 submissions, 38 were desk-rejected for various reasons including formatting problems, length problems and violation of the dual submission policy. In spite of the record number of submissions, we managed to maintain a similar acceptance rate as past NLP conferences given the vast amount of space available to us at the AsiaWorld-Expo. In the end, we accepted 683 submissions. Some statistics of the accepted papers can be found below.


\begin{table}[h]
    \begin{center}
\begin{tabular}[h]{|c|c|c|c|c|}
\hline
    & \textbf{Long Papers} & \textbf{Short Papers} & \textbf{Total} \\
\hline
    \textbf{Reviewed} & 1813 & 1063 & 2876 \\
\hline
    \textbf{Accepted as Oral} & 164 &  48 & 212 \\
\hline
    \textbf{Accepted as Poster} & 301 & 170 & 471 \\
\hline
    \textbf{Total Accepted} & 465 (25.6\%) &  218 (20.5\%) & 683 (23.7\%) \\
\hline
\end{tabular}
    \end{center}
\end{table}

In addition, EMNLP-IJCNLP 2019 will feature 11 papers accepted by the Transactions of the Association for Computational Linguistics (TACL), out of which 8 will be presented as orals and 3 as posters.		
 
Handling close to 3000 submissions was a daunting task, but we were fortunate that a large team of volunteers from our community offered to help. Unlike last year’s EMNLP, where a submission was reviewed in one of eight mega-areas, we organized this year’s submissions into 18 areas, hoping that smaller areas would make things more manageable for our program committee. While traditionally large areas such as Information Extraction, Machine Learning for NLP, and Machine Translation and Multilinguality continued to receive a large number of submissions, areas such as Dialog and Interactive Systems and Summarization and Generation have grown significantly owing to the recent surge of interest in automated response generation.

We adopted a program committee structure similar to that of ACL 2019. For each area, we invited one Senior Area Chair, who worked with a team of Area Chairs (ranging from 4 to 18 per area) and an army of reviewers (1721 in total across all areas). Having a large number of ACs (152 in total) allowed us to assign each of them a reasonable number of papers, which in turn enabled them to better focus on evaluating each paper. Each submission was assigned to three reviewers and one AC. We allowed both the reviewers and the ACs to bid for papers, but used a combination of their bids and the TPMS (Toronto Paper Matching System) scores to assign papers. Although this lengthened the paper assignment process, we believe it allowed us to better match the submissions with reviewers. We also adopted a review form similar to what was used in ACL 2019 as we heard generally good feedback about less structured review forms. While NAACL HLT 2019 and ACL 2019 eliminated author response, we decided that it would be beneficial to keep it even though it put time pressure on our already tight reviewing schedule and resulted in additional work for our program committee members.

This year we received some submissions that raised ethical concerns from the reviewers, and we found that no existing guidelines could be applied. We decided to err on the side of acceptance, encouraging authors of otherwise acceptance-worthy papers to more deeply explore these issues in final drafts, and encouraging the community to carry out further work.

We are extremely grateful to all the Senior Area Chairs, especially those who had a large number of submissions in their areas. The Senior Area Chairs did a fantastic job in nominating Area Chairs, recruiting reviewers and making final recommendations. We would also like to thank all the Area Chairs and reviewers for their hard work in writing meta-reviews and reviews, as well as leading and participating in the discussions. Special thanks to those emergency reviewers who offered help with short notice. Without the dedication of our program committee members, we would not be able to put together this conference program.

Award papers are an integral part of every NLP conference. Based on recommendations made by the ACs and the reviewers, we identified five candidates for the Best Paper award and another five for the Best Resource Paper award. We would like to thank Tim Baldwin, Dan Gildea, Qun Liu, Ellen Riloff, and Luke Zettlemoyer for serving in the Best Paper award committee, and Katrin Erk, Graeme Hirst, Gina-Anne Levow, Percy Liang, and Nianwen Xue for serving in the Best Resource Paper award committee. The award winners will be announced at the closing ceremony.

We are excited to have the following three keynote speakers: Noam Slonim (IBM Haifa), on automated debating technologies; Meeyoung Cha (KAIST), on research challenges in computational social science; and Kyunghyun Cho (NYU), on neural sequence modeling. We would like to thank them for traveling to Hong Kong to give the keynote speeches.

There are also many other people who contributed tremendously to the conference program, and we are very grateful for their help:

\begin{itemize}
   \item Kentaro Inui, the General Conference Chair, who is always there to offer his help and advice;
   \item All the members of the Conference Coordinating Committee, who provided valuable advice on various issues that came up during the review process;
   \item David Chang and Julia Hockenmaier, Program Chairs of EMNLP 2018, who shared very helpful tips from their past experience;
   \item Rich Gerber from Softconf, who helped us set up the conference submission site and always responded to our queries promptly;
   \item Other recent *ACL chairs who offered their help when we contacted them despite their busy schedule;
   \item TACL editors-in-chief Mark Johnson, Lillian Lee and Brian Roark, as well as TACL editorial assistant Cindy Robinson, for coordinating the TACL presentations with us;
   \item Micha Elsner, Fei Liu and Pontus Stenetorp, the Publication Chairs, who worked hard to compile the conference proceedings and kindly accommodated many last minute requests from authors;
   \item Kevin Duh, Henning Wachsmuth, Wei Xu and Sebastian Ruder, the Website Chairs and Publicity Chairs, who helped us make numerous announcements in a timely manner;
   \item Serena Villata and Kai-Wei Chang for preparing the conference handbook;
   \item Members of the Best Paper Award Committee for spending time reading and evaluating the best paper candidates;
   \item Members of the Local Organizing Committee for making the local arrangements;
   \item Derek Wong, the Remote Presentation Chair, for taking care of remote presentations;
   \item Priscilla Rasmussen, whom we directed many inquiries to.
   
\end{itemize}

Again, welcome to EMNLP-IJCNLP 2019! We hope you will have a memorable conference experience!

\vspace{3em}

\noindent EMNLP-IJCNLP 2019 Program Co-Chairs \\

\noindent \textit{Jing Jiang}, Singapore Management University, Singapore \\
\noindent \textit{Vincent Ng}, University of Texas at Dallas, USA\\
\noindent \textit{Xiaojun Wan}, Peking University, China

\index{Jiang, Jing}
\index{Ng, Vincent}
\index{Wan, Xiaojun}
